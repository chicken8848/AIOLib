\documentclass{article}
\usepackage{graphicx}
\graphicspath{./diagrams}
\usepackage{mathtools}
\usepackage{amsfonts}
\usepackage{amssymb}
\usepackage{amsmath}
\usepackage{amsthm}
\usepackage{physics}
\usepackage{pgfplots}
\pgfplotsset{compat=1.18}

\newtheorem{definition}{Definition}[section]

\title{Algorithm Design Manual References}
\author{Joshua John Lee Shi Kai}

\begin{document}
\section{Algorithm Analysis}
\subsection{Common Summation Formulae}
\textit{Sum of a power of integers} | From the big picture perspective, the important thing is that
the sum is quadratic, not that the constant is $\frac{1}{2}$. In general,

\begin{equation*}
	S(n,p) = \displaystyle\sum_{i=1}^{n} i^p = \Theta (n^{p+1})
\end{equation*}
for $p \geq 0$. Thus, the sum of squares is cubic, and the sum of cubes is quartic.

\noindent \\ For $p < -1$, this sum $S(n,p)$ always converges to a constant as $n \rightarrow \infty$,
while for $p \geq 0$ it diverges. The interesting case between these is the harmonic numbers,
$H(n) = \sum_{i=1}^{n} \frac{1}{i} = \Theta (log n)$.

\noindent \\ \textit{Sum of a geometric progression} | In geometric progressions, the index of the
loop affects the exponent, that is,
\begin{equation*}
	G(n,a) = \displaystyle\sum_{i=0}^{n} a^i = \frac{(a^{n+1} - 1)}{(a-1)}
\end{equation*}
How we interpret this sum depends upon the \textit{base} of the progression, in this case \textit{a}.
When $|a| < 1, G(n, a)$ converges to a constant as $n \rightarrow \infty$.

\noindent \\ This series convergence proves to be the great "free lunch" of algorithm analysis. It means
that the sum of a linear number of things can be constant, not linear. For example, $1+\frac{1}{2} + \frac{1}{4}
	+ \frac{1}{8} + \dots \leq 2$ no matter how many terms we add up.

When $a > 1$, the sum grows rapidly with each new term, as in $1 + 2 +  4 + 8 + 16 + 32 = 63$. Indeed,
$G(n,a) = \Theta (a^{n+1}) \text{ for } a > 1$.

\end{document}
