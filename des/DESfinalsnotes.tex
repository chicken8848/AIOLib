\documentclass{article}

\usepackage{amsfonts}
\usepackage{amssymb}
\usepackage{amsmath}
\usepackage{amsthm}
\usepackage{physics}
\usepackage{pgfplots}
\pgfplotsset{compat=1.18}
\usepackage[a4paper, total={6in, 8in}]{geometry}

\title{DES Notes for the final in week 12}
\author{Joshua John Lee Shi Kai}
\begin{document}
\maketitle
\tableofcontents
\newpage
\section*{Introduction}

\begin{enumerate}
	\item Energy source
	      \begin{itemize}
		      \item Sun
	      \end{itemize}
	\item Solar Panel
	      \begin{itemize}
		      \item Parameters
		      \item Sizing
	      \end{itemize}
	\item Solar Charge Controller
	\item Battery
	      \begin{itemize}
		      \item Parameters
		      \item Type
		      \item Sizing
	      \end{itemize}
	\item LED
	      \begin{itemize}
		      \item Load
	      \end{itemize}
\end{enumerate}

\section{Week 6}
\subsection{Solar Radiation Spectrum}
The sun emits EM radiation across most of the electromagnetic spectrum.
\begin{itemize}
	\item \textbf{Spectral irradiance} is the irradiance of a \textit{surface}
	      at a given wavelength.
	\item \textbf{Irradiance} is the power received per unit area
\end{itemize}
Irradiance is a measure of solar power per unit area at any moment
in time. Measured in $W/m^2$ \\
Irradiation ($Wh/m^2$) is the total amount of solar energy per unit
are received over a given period.\\
Peak sun hour is equivalent to the total irradiation
received at the site during the day. 1 peak sun hour is equivalent to
$1kWh/m^2$

\subsection{Solar Irradiance Variation}
Solar irradiation variation depends on location on globe and the position
of sun throughout.
Here are some considerations for the Energy/Power available for
collection:
\begin{enumerate}
	\item Weather and climate
	\item Tilt angle and orientation
	\item Shadowing due to surrounding buildings, vegetation
\end{enumerate}
\subsection{Solar Module}
Solar irradiation proveds energy in the form of a photon. The solar module
(photvoltaic panel, solar panel) absorbed the photon to produce electron
hole pair that conduct electricity.
\subsubsection{Formation of Electron-Hole Pairs with Solar Energy}
Semiconductors (p and n) have a bandgap $< ~5eV$. \\
Attaching a piece of silicon to a circuit will not make current flow
however, because the electrons in the conduction band will move in to
occupy vacant sites in the valance band. We have to create a driving
force. A driving force has to be present to move the electrons created
out of the semiconductor.
\subsubsection{Putting p-n doped semicon together}
\begin{center}
	\begin{tabular}[c]{l|l|l}
		\hline
		                     & p-doped   & n-doped   \\
		\hline
		Major charge carrier & holes     & electrons \\
		Minor charge carrier & electrons & holes     \\
		Overall charge       & Neutral   & Neutral   \\

		\hline
	\end{tabular}
\end{center}
\subsubsection{Formation of $+$ and $-$ Terminals (Driving Force)}
So when put together, the electrons will move from n to p leaving behind
the $+$ ion and creating $-$ ion. So when there is photon with energy $>$
the bandgap. The electron will move from $p \rightarrow n$ junction.
Resulting in current. This current is called \textbf{photocurrent} $I_{sc}$.

\section{Week 8}
\subsubsection{Current in a p-n junction}
Electric field goes from $n \rightarrow p$. $I_{D}$ or \textbf{drift current}
goes in the opposite direction to the electric field. and can be modelled as
\begin{equation}
	I_{D} = I_{o}(e^{\frac{qV}{k_{B}T}}-1)
\end{equation}
Where:
\begin{itemize}
	\item $I_o$: Reverse saturation current
	\item $q$: charge of an electron
	\item $k_{B}$: Boltzmann constant
	\item $T$: Temperature in Kelvin
\end{itemize}
\begin{center}
	\begin{tabular}{l|l|l}
		Name                 & Symbol    & Description                                           \\
		\hline
		Photocurrent         & $I_{sc}$  & Current during a short circuit when the resistance is \\
		                     &           & zero. Maximum current output of the PV cell.          \\
		\hline
		Open-circuit voltage & $V_{oc}$  & Voltage during an open circuit when resistance is at  \\
		                     &           & maximum. Maximum voltage output of the PV cell.       \\
		\hline
		Maximum power        & $P_{max}$ & Maximum power output of the PV cell.                  \\
		\hline
		Maximum power point  & MPP       & The point on the I-V curve at which the PV cell will  \\
		                     &           & generate the most power                               \\
		\hline
		Voltage at MPP       & $V_{mp}$  & The PV cell voltage at MPP                            \\
		\hline
		Current at MPP       & $I_{mp}$  & The PV cell current at MPP                            \\
		\hline
	\end{tabular}
\end{center}
\subsubsection{Photocurrent}
The effect of the illumination of light is the development of photogenerated
carriers(electron-hole) pair. The electric field causes the photegenerated
carries to flow
\begin{itemize}
	\item Electrons to the n
	\item Holes to the p
\end{itemize}
The photocurrent produced is in the opposite direction of $I_{D}$. \\
Well then the ouput current is simply
\begin{align}
	I & = I_{D} - I_{sc}                                        \\
	  & \Rightarrow I = I_{o}(e^{\frac{qV}{k_{B}T}}-1) - I_{sc}
\end{align}
When the solar cell is in operation, $I_{sc} > I_{o}(e^{\frac{qV}{k_{B}T}}-1)$
and current flows from the p terminal. \\
The equation for the IV plot can be written as follows
\begin{equation}
	I_{r} = I_{sc} - I_{o}(e^{\frac{qV}{k_{B}T}}-1)
\end{equation}
\subsubsection{Open Circuit Voltage}
If the terminals of the solar cells are not connected, no current flows.
The voltage that develops across the device is called the open circuit
voltage $V = V_{oc}$
\begin{equation}
	V_{oc} = \frac{k_{B}T}{q}ln(\frac{I_{sc}}{I_o}+1)
\end{equation}
\subsubsection{Output Power}
Output power is determined experimentally. Use graph to determine
$P_{max}$. Geometrically, it can be found by maximizing the area of the
green rectangle that can fit within the I-V plot of the solar cell.
\begin{gather}
	P_{max} = I_{max} V_{max} = FF I_{sc}V_{oc} \\
	FF = fill factor = \frac{P_{max}}{I_{sc}V_{oc}}
\end{gather}
\subsubsection{Efficiency}
\begin{equation}
	\eta = FF \frac{I_{sc}V_{oc}}{P_{in}}
\end{equation}
\section{Week 9}
\subsection{Capacity of Battery}
Energy stored in a battery is in $Wh$ but batteries are rated in terms
of capacity $Ah$. Capacity ($Ah$) measures the amount of charge in a fully
charged battery that can be delivered under specified or nominal discharge
current and temperature. It is simply the product of the current drawn from
a battery, multiplied by the number of hours this current flows.

\noindent \\ How long can a 2Ah battery discharge a current at 1A? 2 hours.

\subsection{Depth of Discharge (DOD)}
Percentage of the battery that has been discharge relative to the overall
capacity of the battery.
\begin{equation}
	DOD = \displaystyle\frac{\text{capacity that is discharged from a fully charged battery}}{\text{battery capacity}}
\end{equation}
DOD affects the cycle service life of the battery.
For some batteries, there are maximum depth of discharge recommendations.
\subsection{Nominal Voltage}
Nominal voltage or mid-point voltage is the voltage when 50\% of the
available capacity is discharged.
\subsection{Cut-off Voltage}
Minimum allowable voltage to prevent damage to the battery.
\subsubsection{Discharge Current Effect on Voltage and Capacity}
Voltage of the battery is different with different discharge current.
The \textbf{higher} the discharge current, the \textbf{lower} the voltage.
Capacity changes with the value of the discharge current. The \textbf{lower}
the discharge current, the \textbf{longer} the discharge time.
\subsection{Specific Energy/Energy Density}
Specific energy ($\frac{Wh}{kg}$) is the energy stored per unit mass.
Energy density ($\frac{Wh}{L}$) is the energy stored per unit volume
Energy content ($Wh$) of a battery determines its run time; related
to the specific energy or energy density.
The theoretical limit of specific energy is determined by standard
reduction potential (voltage of the battery), the number of electrons involved,
and mass of th eactive materials.
\begin{equation}
	\text{Theoretical specific energy of battery}[\frac{Wh}{kg}] = \displaystyle\frac{V[V]\times(\text{number of electrons}\times q)[Ah]}{\text{mass of active materials}[kg]}
\end{equation}
\subsection{Power Density and Specific Power}
Power density ($\frac{W}{L}$) is the maximum amount of power per unit volume.
Specific power ($\frac{W}{kg}$) is the maximum amount of power per unit
mass. Which is a characteristic of the battery chemistry and packaging.
\subsection{Self Discharge Rate}
Loss of capacity through reactions during the battery storage and when battery
is not connected to a load. Self-discharge is usually reversible when a
battery is recharged (the capacity is restored). For lead-acid batteries
the product of discharge reaction is lead sulfate which maya build and grow
irreversibly if the recharge doesn't occur within a certain amount of time.
\subsection{End of Life (EOL)}
Batteries that have reached the end of their usefulness and/or lifespan
and no longer operate at sufficient capacity. Use graph find curve.
\subsection{Days of Autonomy}
Days of autonomy is the duration the battery can supply the site's loads
without any support from generation sources. (eg, cloudy days). System
design requirement that takes the following into consideration,
\begin{itemize}
	\item System application - critical load applications require more autonomy
	      than non-critical appliactions
	\item System availability - minimum percentage of the time that the PV system
	      should be able to satisfy the system's specified design loads. Standalone
	      PV system requires some battery reserve to ensure reliability of service
	      and to provide time for intervention in the event of an unanticipated occurence.
\end{itemize}
\subsection{Designing a Standalone PV System}
\subsubsection{Determine the load energy requirement}
\begin{enumerate}
	\item Assume total system losses
	\item Determine peak sun hours
	\item Decide on Array:Load ratio (More means less chance for standalone system to not work)
	\item Choose a PV module
	\item Daily photovoltaic Ah needed: load requirement x A:L ratio
	\item Convert total system losses to percentage and subtract from 1
	\item Daily photovoltaic ampere-hours per chosen solar panel: system loss percentage x peak sun hours x current at $P_{max}$
	\item Number of panels required: Daily photovoltaic ampere-hours needed / daily photovoltaic ampere-hours per chosen solar panel
	\item Actual number of panels required: Round number of panel required up
	\item Go to sg.rs-online.com
\end{enumerate}
\subsubsection{Sizing of the battery}
\begin{enumerate}
	\item Nominal system voltage: Refer to Load
	\item Days of autonomy: Determine by use-case
	\item Total daily load: Load current [A] x hour of operations per day [h] $\rightarrow$ [Ah]
	\item Unadjusted battery capacity: Days of autonomy x total daily load [Ah]
	\item Maximum allowable depth of discharge: based on battery, 80\% for lead acid so it won't die
	\item Capacity adjusted for MDOD: Unadjusted battery capacity / MDOD [Ah]
	\item Maximum daily depth of discharge: Based on EOL then look at graph
	\item Capacity adjusted MDDOD: Total daily load / MDDOD
	\item Percent of capacity at EOL: set by how you want to sell the battery
	\item Capacity adjusted for EOL: Capacity adjusted for MDOD / Percent of capacity at EOL
	\item Capacity adjusted for DODs or EOL: Greatest capacity adjusted value
	\item Design margin factor: anything $\geq 1$
	\item Capacity adjusted for design margin: greatest value x design margin factor
	\item Find battery on sg.rs-online.com
\end{enumerate}
\end{document}
