\documentclass{article}
\usepackage{verse}
\usepackage{bibleref}
\usepackage{attrib}
\usepackage[a4paper, total={6in, 8in}]{geometry}

% Custom Macros
\def\V#1 {\(^{#1}\)}
\def\vattrib#1#2{\attrib{\bibleverse{#1}(#2)}}

\title{Bible Readings in Oikos}
\author{Joshua's Notes}

\begin{document}
\maketitle
\section*{Introduction} % (fold)
\label{sec:introduction}
Mark is a very succint, quick, gospel. He focuses on Jesus as a worker, the deeds of Jesus. That's the
significance of this book. Mark was not part of the 12 disciples. He was a young boy during the life
and ministry of Jesus. The early church of Jerusalem met in Mark's home. He accompanied the apostle
Paul and Barnabas on the first part of their missionary outreach. Mark later became a sort of assistant
to the apostle Peter. And most of these events were an eyewitness testimony of Peter, because Mark
was so young during the time.

\section{The promise of faith}
\subsection{Genesis 35}
\begin{verse}
	\V{9} God appeared to Jacob again, when he came from Paddan-aram, and blessed him. \V{10} And God
	said to him,
	\begin{quote}
		"Your name is Jacob; no longer shall your name be called Jacob, but Israel shall be your name."
	\end{quote}
	So he called his name Israel \V{11} And God said to him,
	\begin{quote}
		"I am God Almighty. be fruitful and multiply.
		A nation and a company of nations shall come from you, and kings shall come from your own body. \V{12} The land
		that I gave to Abraham and Isaac I will give to you, and I will give the land to your offspring after you."
	\end{quote}
	\V{13} Then God went up from him in the place where he had spoken with him. \\
	\vattrib{Genesis}{35:9-13}
\end{verse}
\subsection{Hebrews 11:11-20}
\subsubsection{The Promise}
\begin{verse}
	\V{11} By faith Sarah herself received power to conceive, even when she was past the age, since she
	considered him faithful who had promised.
	\V{12} Therefore from one man, and him as good as head,
	were born descendants as many as the stars of heaven and as many as the innumerable grains of
	sand by the seashore. \\
	\vattrib{Hebrews}{11:11-12}
\end{verse}
\subsubsection{You might not see it}
\begin{verse}
	\V{13} These all died in faith, not having received the things promised, but having seen them and
	greeted them from afar, and having acknowledged that they were strangers and exiles on the earth. \\
	\vattrib{Hebrews}{11:13}
\end{verse}
\subsubsection{An eternal land}
\begin{verse}
	\V{14} For people who speak thus make it clear that they are seeking a homeland. \V{15} If they had
	been thinking of that land from which they had gone out, they would have had opportunity to return.
	\V{16} But as it is, they desire a better country, that is, a heavenly one. Therefore God is not
	ashamed to be called their God, for he has prepared for them a city. \\
	\vattrib{Hebrews}{11:11-20}
\end{verse}
\subsubsection{The test}
\begin{verse}
	\V{17} By faith Abraham, when he was tested, offered up Isaac, and he who had received the promises
	was in the act of offering up his only son, \V{18} of whom it was said,
	\begin{quote}
		"Through Isaac shall your offspring be named."
	\end{quote}
	\V{19} He considered that God was able even to raise him from the dead, from which, figuratively speaking,
	he did receive him back. \V{20} By faith Isaac invoked future blessings on Jacob and Esau.
\end{verse}
\section{Edom will be humbled}
\subsection{The Violence of Esau}
\begin{verse}
	\V{11} The sons of Eliphaz were Teman, Omar, Zepho, Gatam, and Kenaz. \V{12} (Timna was a concubine
	of Eliphaz, Esau's son; she bore Amalek to Eliphaz.)\\
	\vattrib{Genesis}{36:11-12}
\end{verse}
\subsection{The vision of Obadiah}
\begin{verse}
	\V{1} Thus says the Lord God concerning Edom: We have heard a report from the Lord,
	and a messenger has been sent among the nations:
	\begin{quote}
		“Rise up! Let us rise against her for battle!”
	\end{quote}
	\V{2} Behold, I will make you small among the nations;
	you shall be utterly despised.
	\V{3} The pride of your heart has deceived you,
	you who live in the clefts of the rock,
	in your lofty dwelling,
	who say in your heart,
	\begin{quote}
		“Who will bring me down to the ground?”
	\end{quote}
	\begin{quote}
		"\V{4} Though you soar aloft like the eagle,
		though your nest is set among the stars,
		from there I will bring you down,"
	\end{quote}
	declares the Lord.
\end{verse}
But why does the Lord despise Edom?
\begin{verse}
	\V{10} Because of the violence done to your brother Jacob, shame shall cover you, \\
	and you shall be cut off forever.
\end{verse}
\subsubsection{The coming of the Lord}
\begin{verse}
	\V{15} For the day of the Lord is near upon all the nations. As you have done, it shall be done to you;
	your deeds shall return on your own head. \\
	\vattrib{Obadiah}{1:15}
\end{verse}
\subsubsection{The final victory of the Lord's Kingdom}
\begin{verse}
	\V{21} Saviors shall go up to Moutn Zion to rule Moutn Esau, and the kingdom shall be the Lord's.\\
	\vattrib{Obadiah}{1:21}
\end{verse}

\end{document}
