\documentclass{article}
\usepackage{verse}
\usepackage{bibleref}
\usepackage{attrib}

% Custom Macros
\def\V#1 {\(^{#1}\)}
\def\vattrib#1#2{\attrib{\bibleverse{#1}(#2)}}

\title{The Gospel of Mark}
\author{Joshua's Notes}

\begin{document}
\maketitle
\section*{Introduction} % (fold)
\label{sec:introduction}
Mark is a very succint, quick, gospel. He focuses on Jesus as a worker, the deeds of Jesus. That's the
significance of this book. Mark was not part of the 12 disciples. He was a young boy during the life
and ministry of Jesus. The early church of Jerusalem met in Mark's home. He accompanied the apostle
Paul and Barnabas on the first part of their missionary outreach. Mark later became a sort of assistant
to the apostle Peter. And most of these events were an eyewitness testimony of Peter, because Mark
was so young during the time.

\section{Chapter 1}
\subsection{John Appeared, Baptizing in the Wilderness}
\begin{verse}
	\begin{altverse}
		\V{1}
		The beginning of the gospel of Jesus Christ, the Son of God. \\
	\end{altverse}
	\vattrib{Mark}{1:1}
\end{verse}
This gospel of which we are concerned, is the only everlasting good news. Without Jesus, our mentality
becomes very materialistic, happiness becomes
the end. There would be no meaning to life, and everyone tries to be happy, so they will end up doing
bad stuff.

\begin{verse}
	\V{2}
	As it is written in Isaiah the prophet,
	\begin{quote}
		"Behold, I send my messenger before your face, \\
		Who will prepare your way, \\
		\V{3} The voice crying in the wilderness:\\
		Prepare the way of the Lord,\\
		Make his paths straight,"
	\end{quote}
	\vattrib{Mark}{1:2}
\end{verse}
Mark is referring to the prophecy given by Isaiah.
\begin{verse}
	\V{1} Comfort, comfort my people, says your God. \\
	\V{2} Speak tenderly to Jerusalem, \\
	and cry to her \\
	that her warfare is ended, \\
	that her iniquity is pardoned, \\
	that she has received from the Lord's hand \\
	double for all her sins. \\
	\V{3} A voice cries: \\
	In the wilderness prepare the way of the Lord; \\
	make straight in the desert a highway for our God.\\
	\vattrib{Isaiah}{1:1-3}
\end{verse}
This is one of many Old Testament prophecies that actually has a dual fulfillement in that this prophecy
relates to the first coming of Jesus, and also the second coming of Christ. And that is many case with
many Old Testament prophecies. Mark is pointing out the fact that Isaiah prophesied of one who would
come and prepare the way. And we're told who that is.

\begin{verse}
	\V{4} John appeared, baptizing in the wilderness and proclaiming a baptism of repentance for the
	forgiveness of sins. \V{5} And al the country of Judea and all Jerusalem were going out to him
	and were being baptized by him in the river Jordan, confessing their sins. \\
	\vattrib{Mark}{1:4-5}
\end{verse}
This is a huge move of God, because people were gathering, for one purpose: to confess their sins.
It's not easy to get people to get together to talk about their sin, even more so repenting of it.
It's a miracle! What does it mean to repent though? Repent means to have a change of mind, to change
your actions based on your change of mind. What is this move of God going to do? How does repentance
prepare the way of the Lord? We have to go to another gospel account to answer this question.

\subsubsection{Why is repenting important?}
\begin{verse}
	\V{29} When all the people heard this, and the tax collectors too, they declared God just, having
	been baptized with the baptism of John, \V{30} but the Pharisees and the lawyers rejecetd the
	purpose of God for themselves, not having been baptized by him. \\
	\vattrib{Luke}{7:29-30}
\end{verse}
The power of repentance is that it opens our hearts to hear God, and respond to him in a positive
manner. If there is a hardness of heart that refuses to repent, the Word of God just kind of bounces
off us. The people who received Jesus and understood him, they prepared by repenting.

\subsection{The Spirit of Elijah}
\begin{verse}
	\V{6} Now John was clothed with camel's hair and wore a leather belt around his waist and ate
	locusts and wild honey. \\
	\vattrib{Mark}{1:6}
\end{verse}
Elijah wore much of the same clothes. But more interesting than that is that, in Luke, when the
angel prophecied to Zechariah about John the baptist, that John would go forth in the spirit and
power of Elijah. But John never performed a miracle, yet the Bible tells us he goes forth in the
spirit and power of Elijah, which is shown in his clothing.

\begin{verse}
	\V{7} And he preached, saying,
	\begin{quote}
		"After me comes he who is mightier than I, \\
		the strap of whose sandals I am not worthy to stoop down and untie."
	\end{quote}
	\vattrib{Mark}{1:7}
\end{verse}
To the jew particular statement elevated Jesus and humbled John. The teachers in Judaism used to say that
a teacher could ask pretty much anything of his student except to help him off with his shoes. That
was supposed to be beneath any student. John comes along and to speak of the greatness of the one
who comes after and says I am not worthy to take off his shoes. He further goes on to contrast his
ministry and continues with verse 8.

\begin{verse}
	\begin{quote}
		\V{8} "I have baptized you with water, but he will baptize you with the Holy Spirit."
	\end{quote}
	\vattrib{Mark}{1:8}
\end{verse}
John is immersing you in water, and he's preparing you to meet Jesus. But Jesus is going to immerse
you in God. Jesus is going to bring God into your life. Baptism doesn't change your life, the spirit
of God does. Living in him. Baptism won't change the condition of your heart, but being filled with
the Holy Spirit will.
\subsection{The Baptism of Christ}
\begin{verse}
	\V{9} In those days Jesus came from Nazareth of Galilee and was baptized by John in the Jordan.
	\V{10} And when he came up out of the water, immediately he saw the heavens being torn open and the
	Spirit descending on him like a dove. \V{11} And a voice came from heaven, "You are my beloved Son;
	with you I am well pleased." \\
	\vattrib{Mark}{1:9-11}
\end{verse}
Jesus came to the midst of the Jordan river filled with sinful people, and identified with them.
Jesus would hang on the cross after 3 years for these sinners. The bible says he became sin, that
we might become the righteousness of God.
\subsection{Driven into the Wilderness}
\begin{verse}
	\V{12} The Spirit immediately drove him out into the wilderness. \V{13} And he was in the wilderness
	forty days, being tempted by Satan. And he was with the wild animals, and the angels were ministering
	to him. \\
	\vattrib{Mark}{1:12}
\end{verse}
Jesus had 40 days of temptation, even though we only learn of 3 particular temptations. Imagine being
tempted for 40 days. This is Jesus going through what we go through. Most of our lives we have to deal
with temptation and the enemy trying to bring us further from God. Jesus came to drink the entire
cup of the condition of man.
\subsubsection{The sympathy of Jesus}
\begin{verse}
	\V{15} For we do not have a high priest who is unable to sympathize with our weaknesses, but one
	who in every respect has been tempted as we aree, yet without sin.
\end{verse}
\vattrib{Hebrews}{4:15}
He not only experienced what we experienced, but also gave the way not to fall into temptation. With
the word of God, and faith, he came through with victory. Which can now become our victory. To be victorious
over sin, its not a method, we look to Jesus. We enter into his victory. The source is a person, not
a few steps. (As we see on facebook or whatever)

\subsection{Jesus Proclaims the Gospel}
\begin{verse}
	\V{14} Now after John was arrested, Jesus came into Galilee, proclaiming the gospel of God, \V{15} and
	saying, "The time is fulfilled, and the kingdom of God is at hand; repent and believe in the gospel." \\
	\vattrib{Mark}{1:14-15}
\end{verse}
Jesus spoke repentance as well. He also said the time was fulfilled, the coming of Messiah!
\subsection{Follow Jesus}
\begin{verse}
	\V{16} Passing alongside the Sea of Galilee, he saw Simon and Andrew the brother of Simon casting
	a net into the sea, for they were fishermen. \V{17} And Jesus said to them, "Follow me, and I will
	make you become fishers of men." \V{18} And immediately they left their nets and followed him. \\
	\vattrib{Mark}{1:16-18}
\end{verse}
This is actually the second time Jesus has spoken to them, and Mark does not record the first time.
But now that he has started his Galilean ministry, he called them and they immediately followd him.
And Mark loves to use this word "immediately", uses it about 40 times. Peter had a family, and they
dropped everything and immediately followed Jesus, even though he was probably the only breadwinner.
When Jesus calls us, its nay or yay. Its impossible to do both. Its a decision, are you going to follow
Jesus?
\begin{verse}
	\V{19} And going on a little farther, he saw James the son of Zebedee and John his brother, who were in
	their boat mending the nets. \V{20} And immediately he called them, and they left their father
	Zebedee in the boat with the hired servants and followed him.
\end{verse}
\vattrib{Mark}{1:19-20}
And he says, "I'm going to make you fishers of men." We all talents, what he's saying is take the
things he has given you and follow him to the kingdom of God. These guys were fishermen, Jesus said
"Fine, let's go fish, but we're gonna fish for different things. You're a bulider, let's go build
different things. You're a banker, let's go invest in different things." What you are in the world,
bring it into the kingdom, that has eternal scope and value. New goals for old powers.
\subsection{The Authority Of Jesus}
\begin{verse}
	\V{21} And they went into Capernaum, and immediately on the Sabbath he entered the synagogue and
	was teaching. \V{22} And they were astonished at his teaching, for he taught them as one who had
	authority, and not as the scribes. \\
	\vattrib{Mark}{1:21-22}
\end{verse}
The Jewish teachers back in those days, they would spend a lot of time quoting other teachers. And
sometimes give opposing viewpoints of different Rabbis, but Jesus didn't do that, he just spoke
by the authority of his own word. We forget sometimes when we get into debates, that truth is not a
debate process, truth is a person. When we get into debates, it's important to remember not to get
caught up in the conversation of opinion, but rather let the words of Jesus to speak for itself.
Because Jesus spoke with authority. Don't get lost in other people's and your own opinion. It's not
just important to tell others what Jesus said, it is also important to pinpoint it for others.
\begin{verse}
	\V{23} And immediately there was in their synagogue a man with an unclean spirit. And he cried
	out, \\
	\vattrib{Mark}{1:23}
\end{verse}
The word unclean is a borrowing from an Old Testament concept of ceremonially unclean. But here it
just sort of means possessed by demonic spirit. Mark uses language in the greek, which is similar to
the language Paul uses when he describes a believer who is indwelled by the Holy Spirit. Be possessed
by the Lord's spirit, it will bring you freedom, peace, and eternal life.

\subsubsection{How does one get posssessed by a demon?}
The Bible doesn't really explain it.
Is it possible for me as a Christian, to be possessed by a demon? Stay invested in God's spirit,
the two cannot dwell together. You never even see or hear of the mention of an exorcism for a believer
in the Bible. But rather a testing of faith, and to repent if you sin. You can't blame demonic
possession.
\begin{verse}
	\V{18} We know that everyone who has been born of God does not keep on sinning, but he who was
	born of God protects him, and the evil one does not touch him.
	\vattrib{John}{5:18}
\end{verse}
"He" refers to Jesus. When it comes to believers, its a hands off approach when it comes to the work
of the enemy. That does not mean we won't have spiritual battles, or testing.
\begin{verse}
	\V{8} Be sober-minded; be watchful. Your adversary the devil prowls around like a roaring lion, seeking
	someone to devour. \V{9} Resist him, firm in your faith, knowing that the same kinds of suffering
	are being experienced by your brotherhood throughout the world. \\
	\vattrib{IPeter}{5:8-9}
\end{verse}
So the reality of satanic activity real according scripture.
\begin{verse}
	\V{12} For we do not wrestle against flesh and blood, but against the rulers, against the authorities,
	against the cosmic powers over this present darkness, against the spiritual forces of evil in the
	heavenly places. \\
	\vattrib{Ephesians}{6:12}
\end{verse}
Battles are a lot in life. But as a believer, possession is not in the cards for you
\subsection{The Authority Of Jesus}
\begin{verse}
	\V{24} "What have you to do with us, Jesus of Nazareth? Have you come to destroy us? I know who
	you are | the Holy One of God." \\
	\vattrib{Mark}{1:24}
\end{verse}
The demons would often speak through the host to expose Jesus, and then start freaking out. Jesus
always commanded to be silent, not to tell who he was. But why? The Jews wanted to be free of Roman
leadership, Jesus knew that the Zealots would jump at the opportunity to go against the Roman
Empire.
\begin{verse}
	\V{25} But Jesus rebuked him saying,
	\begin{quote}
		"Be silent, and come out of him!"
	\end{quote}
	\V{26} And the unclean spirit, convulsing him and crying out with a loud voice, came out of him. \\
	\vattrib{Mark}{1:26}
\end{verse}
This guy has authority, his exorcism is short. Which is different from other exorcisms.
\begin{verse}
	\V{27} And they were all amazed, so that they questioned among themselves, saying,
	\begin{quote}
		"What is this? A new teaching with authority! He commands even the unclean spirits, and they
		obey him."
	\end{quote}
	\V{28} And at once his fame spread everywhere throughout all the surrounding region of Galilee.\\
	\vattrib{Mark}{1:27-28}
\end{verse}
\begin{verse}
	\V{29} And immediately he left the synagogue and entered the house of Simon and Andrew, with James
	and John. \V{30} Now Simon's mother-in-law lay ill with a fever, and immediately they told him
	about her. \V{31} And he came and took her by the hand and lifted her up, and the fever left her, and
	she began to serve them. \\
	\vattrib{Mark}{1:29-31}
\end{verse}
What does she do with the new life and the healing that was given her? She just starts
serving him. God wants to restore to us what we have lost, and serve him with it. Just like this
woman. What has God restored in your life? Serve the Lord with all of your heart, give it to him.
\begin{verse}
	\V{32} That evening at sundown they brought to him all who were sick or oppressed by demons.
	\V{33} And the whole city was gathered together at the door. \V{34} And he healed many who were
	sick with various diseases, and cast out many demons. And he would not permit the demons to speak,
	because they knew him. \\
	\vattrib{Mark}{1:32-34}
\end{verse}
This is God's compassion, trust in the compassion of God. We do not need to pray him into compassion.
His heart is moved by people with faith in him.
\subsubsection{The Demons know Jesus}
\begin{verse}
	\V{19} You believe that God is one; you do well. Even the demons believe | and shudder!
\end{verse}
You believe in God? Even demons believe in God. Do not rest in the superficiality of knowing Jesus,
or even just believing that he exists. Do not stop at the superficial, it has to be personal.
\begin{verse}
	\V{12} But to all who did receive him, who believed in his name, he gave the right to become children
	of God. \\
	\vattrib{John}{1:12}
\end{verse}
Communion is a picture of receiving Jesus. It is not intellectual, but it is of partaking, and now
it is inside of you. Have you received him as your saviour? It is going past religion, to relationship.
\subsection{The Inconvenience of Following God}
\begin{verse}
	\V{35} And rising very early in the morning, while it was still dark, he departed and went out to
	a desolate place, and there he prayed. \V{36} And Simon and those who were with him searched for him,
	\V{37} and they found him and said to him,
	\begin{quote}
		"Everyone is looking for you."
	\end{quote}
	\V{38} And he said to them,
	\begin{quote}
		"Let us go on to the next towns, that I may preach there also, for that is why I came out."
	\end{quote}
	\V{39} And he went throughout all Galilee, preaching in their \\>
	synangogues and casting out demons. \\
	\vattrib{Mark}{1:35-39}
\end{verse}
A long day of ministry, goes into the night, Jesus probably lays down to get some rest and before
the sun even rose, he went out to pray. This shows his heart toward prayer. Some of us wait for
prayer to be convenient, it is not convenient. It is never convenient to pray. It's hard work,
its challenging, and you may have to give something up to do it. For Jesus, he gave up his sleep,
which we hang onto so tightly. Prayer for Jesus meant sacrifice too. No one likes getting up when
its dark. But let's do it because it needs to be done.
\subsection{Talking to God}
\begin{verse}
	\V{40} And a leper came to him, imploring him, and kneeling said to him, "If you will, you can
	make me clean." \V{41} Moved with pity, he stretched out his hand and touched him and said to him,
	"I will; be clean." \V{42} And immediately the leprosy left him, and he was made clean.
	\vattrib{Mark}{1:40-42}
\end{verse}
When you think about what this man said to Jesus, think about it like a prayer. We don't have to
use eloquent speech. There is an expectation that they have to say it right. Prayer is nothing but
talking, but talking to God. No King James English is okay. Let's dissect this man's prayer.
\begin{enumerate}
	\item It was sincere and desperate | "imploring him"
	\item It was reverent | "and kneeling said to him"
	\item It was humble and submissive | "If you will"
	\item It was believing | "you can"
	\item It acknowledged his need | "MAKE me clean"
	\item It was specific | "make me CLEAN"
	\item It was personal | "make ME clean"
	\item It was brief | just five words in the greek
\end{enumerate}
Five short words and it got God's attention. He's not waiting for you to be flowery, he just wants
a relationship with you and for you to talk to him. We need to get over our American Idol approach to
prayer that elevates one person's prayer above another just because we think they are more spiritual.
That is a worldly attitude, not a biblical one. You need to see people from the perspective of the
kingdom of God. You wanna know who's the greatest? The least among you. Everyone has the same access
to the throne of Grace. The bible says,
\begin{verse}
	\V{16} Therefore, confess your sins to one another and pray for one another, that you may be healed.
	The prayer of a righteous person has great power as it is working.
	\vattrib{James}{5:16}
\end{verse}
And in Jesus we are all made righteous. Because you have made him Lord of your life, and you have
put on the robe of righteousness. So your prayers are powerful and effective.

\noindent \\ What's even more wonderful is that Jesus took pity on the leper, and even touched him,
and healed him! In that day, no one would even greet a leper not mentioning touching them. But Jesus
with such compassion, with such grace, reached down, touched him and healed him. This is God's heart
towards those who are hurting.

\begin{verse}
	\V{43} And Jesus sternly charged him and sent him away at once, \V{44} and said to him,
	\begin{quote}
		"See that you say nothing to anyone, but go, show yourself to the priest and offer for your
		cleansing what Moses commanded, for a proof to them."
	\end{quote}
	\vattrib{Mark}{1:43-44}
\end{verse}
Jesus told this man, in keeping with the law of Moses, to go and have his physical condition
examined by the priest, so that he may be declared clean and be able to enter the temple and worship
with the people. But also communicate with the priest that a miracle had taken place! Not what he did
though.

\subsection{The People Need Jesus}
\begin{verse}
	\V{45} But he went out and began to talk freely about it, and to spread the news, so that Jesus
	could no longer openly enter a town, but was out in desolate places, and people were coming to
	him from every quarter.
\end{verse}
People are still seeking out Jesus. There seems to be a lot of darkness in the world. But there seems
to be some amazing compelling effect of being in darkness, to push you toward the light. And we are
the light of the world. People are drawing to the light of Christ. God intended for you to be alive
today, people need to hear about Jesus Christ. Let your light shine wherever you are.

\section{Chapter 2}
\begin{verse}
	\V{1} And when he returned to Capernaum after some days, it was reported that he was at home. \V{2}
	And many were gathered together, so that there was no more room, not even at the door. And he was
	preaching the word to them. \V{3} And they came, bringing to him a paralytic carried by four men.
	\V{4} And when they could not get near him because of the crowd, they removed the roof above him,
	and when they had made an opening, they let down the bed on which the paralytic lay.
	\vattrib{Mark}{2:1-4}
\end{verse}

\subsection{The Authority Of Jesus}
\begin{verse}
	\V{5} And when Jesus saw their faith, he said to the paralytic, "Son, your sins are forgiven." \V{6}
	Now some of the scribes were sitting there, questioning in their hearts, \V{7} "Why does this man
	speak like that? He is blaspheming! Who can forgive sins but God alone?" \V{8} And immediately
	Jesus, perceiving in his spirit that they thus questioned within themselves, said to them, "Why do
	you question these things in your hearts?" \V{9} Which is easier, to say to the paralytic, "Your
	sins are forgiven, or to say 'Rise, take up your bed and walk'?"
	\vattrib{Mark}{2:5-9}
\end{verse}
To a Jewish mind, this was like a trick question, in their mind, both of these questions are equally
difficult to do. To put yourself in the place of God by saying "Your sins are forgiven" would be
blaspheming. And to say "rise up, take your bed and walk." would be to do a miraculous work of healing.
Jesus was going to prove that he has the authority to do that which is unseen | forgive sins, to do
that which is seen | healing the man.

\begin{verse}
	\V{10} "But you may know that the Son of Man has authority on earth to forgive sins" | he said to
	the paralytic | \V{11} "I say to you, rise, pick up your bed, and go home." \V{12} And he rose
	and immediately picked up his bed and went out before them all, so that they were all amazed and
	glorified God, saying, "We never saw anything like this!"
	\vattrib{Mark}{2:10-11}
\end{verse}
Since Jesus can do what you can see, then he can do what you cannot see, which is say that an
individuals sins are forgiven. And we can do it too, through Jesus.
\subsubsection{Your Sins are Forgiven}
\begin{verse}
	\V{38} Let it be know to you therefore, brothers, that through this man forgiveness of sins is
	proclaimed to you.
	\vattrib{Acts}{13:38}
\end{verse}
We have been given permission to speak on the authority of God's word. To say, "Your sins are forgiven."
\subsection{Levi Called}
\begin{verse}
	\V{13} He went out again beside the sea, and all the crowd was coming to him, and he was
	teaching them. \V{14} And as he passed by, he saw Levi the son of Alphaeus sitting at the tax booth,
	and he said to him, "Follow me." And he rose and followed him.
	\vattrib{Mark}{2:13-14}
\end{verse}
Levi is the same as Matthew, in that he was a Jew, and a tax collector by trade. He was despised
by the Jews, because they collected taxes for Rome | tyrannical governmental structure. Some tax
collectors would also overcharge people, and then pocket the difference. But Jesus says to him,
"Follow me."
\subsection{Jesus came not for the righteous}
\begin{verse}
	\V{15} And as he reclined at table in his house, many tax collectors and sinners were reclining
	with Jesus and his disciples, for there were many who followed him. \V{16} And the scribes of the
	Pharisees, when they saw that he was eating with sinners and tax collectors, said to his disciples,
	"Why does he eat with tax collectors and sinners?" \V{17} And when Jesus heard it, he said to them,
	"Those who are well have no need of a physician, but those who are sick. I came not to call the
	righteous, but sinners." \\
	\vattrib{Mark}{2:15-17}
\end{verse}
It doesn't excuse the behaviour of those he was eating with. But those people were his mission.
Keeping the law got in the way for the Pharisees to accept Jesus, and repent of their sins.

\subsection{Fasting}
\begin{verse}
	\V{18} Now John's disciples and the Pharisees were fasting. And people came and said to him,
	\begin{quote}
		"Why do John's disciples and the disciples of the Pharisees fast, but your disciples do not fast?"
	\end{quote}
	\V{19} And Jesus said to them,
	\begin{quote}
		"Can the wedding guests fast while the bridegroom is with them? As long as they have the
		bridegroom with them, they cannot fast. \V{20} The days will come when the bridegroom is taken
		from them, and then they will fast in that day."
	\end{quote}
	\vattrib{Mark}{2:18-20}
\end{verse}
Jesus is the bridegroom, and we are the bride. Who thinks of fasting during a joyous celebration?
And indeed it was a joyous celebration, the gospel is here. Fasting expressed sorrow, grief, mourning.
In verse 20, I think Jesus is referring to the time between his death and ressurrection.
\subsection{The Old and New Covenants}
\begin{verse}
	\V{21} No one sews a piece of unshrunk cloth on an old garment. If he does, the patch tears away from
	it, the new from the old, and a worse tear is made. \V{22} And no one puts new wine into old wineskins.
	If he does, the wine will burst the skins | and the wine is destroyed, and so are the skins. But new
	wine is for fresh wineskins. \\
	\vattrib{Mark}{2:21-22}
\end{verse}
Christians are not to base their faith on Judaism, although the Law is important. For Christ is the
end of the law for righteousness to everyone who believes.
\subsection{The Sabbath}
\begin{verse}
	\V{23} One Sabbath he was going through the grainfields, and as they made their way, his disciples
	began to pluck heads of grain. \\
	\vattrib{Mark}{2:23}
\end{verse}
To the Jews, that meant harvesting, which was considered to be work on the Sabbath.
\begin{verse}
	\V{24} And the Pharisees were saying to him,
	\begin{quote}
		"Look, why are they doing what is not lawful on the Sabbath?"
	\end{quote}
	\V{25} And he said to them,
	\begin{quote}
		"Have you never read what David did, when he was in need and was hungry, he and those who were
		with him: \V{26} how he entered the house of God, in the time of Abiathar the high priest, and
		ate the bread of the Presence, which is not lawful for any but the priests to eat, and also
		gave it to those who were with him?"
	\end{quote}
	And he said to them,
	\begin{quote}
		"The Sabbath was made for man, not man for the Sabbath. \V{28} So the Son of Man is lord even of
		the Sabbath."
	\end{quote}
	\vattrib{Mark}{2:24-28}
\end{verse}
Human need should take precedence over the ceremonial law. Jesus made the same point about the priests
on the Sabbath with circumcision. The Sabbath was intended to serve the people, not for people to
serve the Sabbath. It should be a blessing to the people.
\begin{verse}
	\V{12} And the Lord said to Moses, \V{13} "You are to speak to the people of Israel and say, 'Above
	all you shall keep my Sabbaths, for this is a sign between me and you throughout the generations,
	that you may know that I, the Lord, sanctify you. \V{14} You shall keep the Sabbath, because it is
	holy for you. Everyone who profanes it shall be put to death. Whoever does any work on it, that soul
	shall be cut off from among his people. \V{15} Six days shall work be done, but the seventh day is
	a Sabbath of solemn rest, holy to the Lord. Whoever does any work on the Sabbath day shall be put
	to death. \V{16} Therefore the people of Israel shall keep the Sabbath, observing the Sabbath
	throughout the generations, as a covenant forever." \\
	\vattrib{Exodus}{31:12-16}
\end{verse}
It was a covenant between God and Israel. It is wisdom to hang out in God's presence and take a day
of rest. But more than that, the Sabbath is part of the ceremonial law, that point to Jesus Christ.
That we should rest in the finished work of the cross. Rest in what Jesus did.
\begin{verse}
	\V{9} So then, there remains a Sabbath rest for the people of God, \V{10} for whoever has entered
	God's rest has also rested from his works as God did from his. \V{11} Let us therefore strive to
	enter that rest, so that no one may fall by the same sort of disobedience. \\
	\vattrib{Hebrews}{4:9-11}
\end{verse}
If you're constantly feeling condemned, you are not resting in his works. That's a works-based salvation.
We do not enter into salvation by our own works. So he says, whoever has entered into God's rest,
has rested from his own works. We who have believed entered that rest, by faith in Jesus, and what
he has accomplished on the cross. Peace and joy are only found in Christ. Having a legalistic
view tends to make one struggle with the belief that they are really saved. But deeds are not a
guarantee of your salvation.

\section{Chapter 3}
\subsection{The Hardness of Heart}
\begin{verse}
	\V{1} Again he entered the synagogue, and a man was there with a withered hand. \V{2} And they
	watched Jesus, to see whether he would heal him on the Sabbath, so that they might accuse him.
	\V{3} And he said to the man with the withered hand, "Come here." \V{4} And he said to them,
	"Is it lawful on the Sabbath to do good or to do hame, to save life or to kill?" But they were
	silent. \\
	\vattrib{Mark}{3:1-4}
\end{verse}
The Jews came to see that keeping the Sabbath as more important than people. And that's what happens
when you get caught up in legalism. It creates a culture of always looking down on people, judging
and condemning because they don't keep the rules.
\begin{verse}
	\V{5} And he looked around at them with anger, grieved at their hardness of heart, and said to the
	man, "Stretch out your hand." He stretched it out, and his hand was restored. \V{6} The Pharisees
	went out and immediately held counsel with the Herodians agains him, how to destroy him. \\
	\vattrib{Mark}{3:5-6}
\end{verse}
This is the hardness of heart of the Pharisees. Upon seeing the goodness of God, they walk out and
try to kill him. What grieves God is hardness of heart. So do not harden your heart against his word.
\begin{verse}
	\V{7} Jesus withdrew with his disciples to the sea, and a great crowd followed, from Galilee and
	Judea \V{8} and Jerusalem and Idumea and from beyond the Jordan and from around Tyre and Sidon.
	When the great crowd heard all that he was doing, they came to him. \V{9} And he told his disciples
	to have a boat ready for him because of the crowd, lest they crush him, \V{10} for he had healed
	many, so that all who had diseases pressed around him to touch him. \V{11} And whenever the unclean
	spirits saw him, they fell down before him and cried out, "You are the Son of God." \V{12} And he
	strictly ordered them not to make him known. \V{13} And he went up on the mountain and called to him
	those whom he desired, and they came to him. \V{14} And he appointed twelve (whom he also named
	apostles) so that they might be with him and he might send them out to preach \V{15} and have
	authority to cast out demons. \V{16} He appointed the twelve: Simon (to whom he gave the name Boanerges,
	that is, Sons of Thunder); \V{18} Andrew, and Philip, and Bartholomew, and Matthew, and Thomas, and
	James the son of Alphaeus, and Thaddaeus, and Simon the Zealot, \V{19} and Judas Iscariot, who
	betrayed him. \\
	\vattrib{Mark}{3:7-19}
\end{verse}
Apostle means one sent forth with authority. Jesus chose them to send them out, and to have authority.
Jesus also chose Judas. There are some people who believe that Judas was born to be a betrayer. But
God's foreknowledge does not dictate the outcome. And we are still the product of our choices. He
made the choice to lie and to steal.
\begin{verse}
	\V{20} Then he went home, and the crowd gathered again, so that they could not even eat. \V{21}
	And when his family heard it, they went out to seize him, for they were saying, "He is out of his
	mind." \\
	\vattrib{Mark}{3:20-21}
\end{verse}
This is how we know that Mary and Joseph had a normal marital relationship. Even Jesus' brother, did
not accept Jesus as the Messiah until he appeared to him after his ressurrection.
\begin{verse}
	\V{36} And a person's enemies will be those of his own household. \\
	\vattrib{Matthew}{10:36}
\end{verse}
Jesus' family also did not seem to support him. He understood what it meant to have a divided home.
\begin{verse}
	\V{22} And the scribes who came down from Jerusalem were saying, "He is possessed by Beelzebul,"
	and "by the prince of demons he casts out the demons." \V{23} And he called them to him and said
	to them in parables, "How can Satan cast out Satan?" \V{24} If a kingdom is divided against itself,
	that kingdom cannot stand. \V{25} And if a house is divided against itself, that house will not be
	able to stand. \V{26} And if Satan has risen up against himself and is divided, he cannot stand,
	but is coming to an end. \V{27} But no one can enter a strong man's house and plunder his goods,
	unless he first binds the strong man. Then indeed he may plunder his house. \V{28} "Truly, I
	say to you, all sins will be forgiven the childen of man, and whatever blasphemies they utter,
	\V{29} but whoever blasphemes against the Holy Spirit never has forgiveness, but is guilty of
	an eternal sin" | \V{30} for they were saying, "He has an unclean spirit." \\
	\vattrib{Mark}{3:22-30}
\end{verse}
This is not a condemnation, but this is all given in the context of the hardness of heart. This is
what happens when the hardness of heart goes so far, that it can no longer be reached. Blaspheming
the Holy Spirit is seeing the goodness, and power of God, but attributing it to the devil. But that
only comes from such a hardness of heart, if you are concerned about blaspheming the Holy Spirit,
you wouldn't even care about blaspheming the Holy Spirit. But why is this sin unforgivable?
\begin{verse}
	\V{8} And when he comes, he will convict the world concerning sin and righteousness and judgement:\\
	\vattrib{John}{16:8}
\end{verse}
The Holy Spirit's role is to convict the world concerning sin. If God convicts you on a daily basis
about the things you are doing wrong, you are made aware of that through the Holy Spirit. If your
heart is so hard, that resist the Holy Spirit, you can never be convicted of your sin.
\begin{verse}
	"You stiff-necked people, uncircumcised in heart and ears, you always resist the Holy Spirit. As
	your fathers did, so do you."
	\vattrib{Acts}{7:51}
\end{verse}

\begin{verse}
	\V{31}	And his mother and his brothers came, and standing outside they sent to him and called him.
	\V{32} And a crowd was sitting around him, and they said to him, "Your mother and your brothers
	are outside, seeking you." \V{33} And he answered them, "Who are my mother and my brothers?" \V{34}
	And looking about at those who sat around him, he said, "Here are my mother and my brothers!" \V{35}
	"For whoever does the will of God, he is my brother and sister and mother."\\
	\vattrib{Mark}{3:31-35}
\end{verse}
This is a reminder, that if you are a Christian, you are born not only as child of God,
but into his family. Look at the people in the church as your family, don't stray away.
This connection in Christ goes on forever.
\section{Chapter 4}
\subsection{To Tremble At God's Word}
\begin{verse}
	\V{1} Again he began to teach beside the sea. And a very large crowd gathered about him, so that he
	got into a boat and sat in it on the sea, and the whole crowed was beside the sea on the land. V{2}
	And he was teaching them many things in parables, and in his teaching he said to them: \V{3} "Listen!
	Behold, a sower went out to sow. \V{4} And as he sowed, some seed fell along the path, and the birds
	came down and devoured it. \V{5} Other seed fell on rocky ground, where it did not have much soil,
	and immediately it sprang up, since it had no depth of soil. \V{6} And when the sun rose, it was
	scorched, and since it had no root, it withered away. \V{7} Other seed fell among thorns, and the
	thorns grew up and choked it, and it yielded no grain. \V{8} And other seeds fell into good soil and
	produced grain, growing up and increasing and yielding thirtyfold and sixtyfold and a hundredfold."
	\V{9} And he said, "He who has ears to hear, let him hear." \V{10} And when he was alone, those around
	him with the twelve asked him about the parables. \V{11} And he said to them, "To you has been given
	the secret of the kingdom of God, but for those outside everything is in parables, \V{12} so that
	\begin{quote}
		"They may indeed see but not perceive, and may indeed hear but not understand, lest they should
		turn and be forgiven.""
	\end{quote}
	\vattrib{Mark}{4:1-12}
\end{verse}
Parables give some sort of spiritual sifting process. To see who is in the inside, and to see who is in
the inside. Those who are on the inside, who understand the scriptures, have humbled themselves before
God. There are many that listen but do not hear, because they don't want to hear. With anyone with a
heart to learn, God responds quickly.
\begin{verse}
	\V{16} But blessed are your eyes, for they see, and your ears, for they hear. \V{17} For truly,
	I say to you, many prophets and righteous people longed to see what you see, and did not see it,
	and to hear what you hear, and did not hear it. \\
	\vattrib{Matthew}{13:16-17}
\end{verse}
The prophets wished to see it, but couldn't because Messiah had not yet arrived. But now Messiah is
here. Pilate asked Jesus, "What is truth?" And Jesus said nothing. For Pilate wasn't wanting
an answer.
\begin{verse}
	\V{13} And he said to them, "Do you not understand this parable? How then will you understand all
	thet parables?" \\
	\vattrib{Mark}{4:13}
\end{verse}
This parable will provide a key to understand all parables. Get this one, and it will give you
greater understanding.
\begin{verse}
	\V{14} The sower sows the word. \V{15} And these are the ones along the path, where the word is
	sown: when they hear, Satan immediately comes and takes away the word that is sown in them.
	\vattrib{Mark}{4:14-15}
\end{verse}
The seed that is being sown is the word of God.
\begin{verse}
	\V{19} When anyone hears the word of the kingdom and does not understand it, the evil one comes and
	snatches away what has been sown in his heart. This is what was sown along the path. \\
	\vattrib{Matthew}{13:19}
\end{verse}
This parable is about the different conditions of your heart. The seed hit a place where there was
no place for it to enter in. There was no place to plant the seed because the heart was hard.
\begin{verse}
	\V{16} And these are the ones sown on rocky ground: the ones who, when they hear the word,
	immediately receive it with joy. \V{17} And they have no root in themselves, but endure for a while;
	then, when tribulation or persecution arises on account of the word, immediately they fall away. \\
	\vattrib{Mark}{4:16-17}
\end{verse}
This is the shallow heart. The seed germinates, but there's no place for the root. Because there's
no place for the root to grow, or depth in their lives for the word of God. They have no place to
hear and store up what has been sown in their lives, and quickly fall away when trouble arrives.
And where does the hardship come from? It is trouble and hardship that comes on account of the word.
It is someone who doesn't want to have inconvenience or discomfort, because they are a believer. In
work, family, friends, people can treat you different because you are a believer.
\begin{verse}
	\V{18} And others are the ones sown among thorns. They are those who hear the word, \V{19} but
	the cares of the world and the deceitfulness of riches and the desires for other things enter in
	and choke the word, and it proves unfruitful. \\
	\vattrib{Mark}{4:18-19}
\end{verse}
This is where the seed doesn't even have a chance to grow. Their hearts are so tied to this world.
The heart of the world are too deeply engrained in their hearts. They hear the word of God, and
their hearts are already too overcrowded from the desires of this world.
\begin{verse}
	\V{20} But those that were sown on the good soil are the ones who hear the word and accept it
	and bear fruit, thirtyfold and sixtyfold and a hundredfold. \\
	\vattrib{Mark}{4:20}
\end{verse}
This is the heart that is open, humble, and teachable. The one that comes to Jesus and says, "I want
to learn, I want to know." There are sometimes that God allows things to happen to us, to change
our heart. It is often when we are at the bottom of our lives that we become open and responsive.
\begin{verse}
	\V{21} And he said to them, "Is a lamp brought in to be put under a basket, or under a bed, and not
	on a stand?" \\
	\vattrib{Mark}{4:21}
\end{verse}
There are many Christians, who cover up the fact that they are Christian. Maybe to avoid standing out,
or trying to blend in. Why am I perhaps hiding the light? Am I embarrassed?
\begin{verse}
	\V{22} "For nothing is hidden except to be made manifest; nor is anything secret except to come
	to light." \\
	\vattrib{Mark}{4:22}
\end{verse}
God's word is meant to be understood. If God's word is hidden to you, its probably a problem on your
end. Its because we are not seeking him. Truth is made to be known.
\begin{verse}
	\V{23} "If anyone has ears to hear, let him hear." \\
	\vattrib{Mark}{4:23}
\end{verse}
What do you need to hear the truth? Ears to hear. How to get ears to hear? You ask God. A child can
understand the scriptures. And they usually do better than adults, because their hearts are open.
\begin{verse}
	\V{8} For everyone who asks receives, and the one who seeks finds, and to the one who knocks it will
	be opened. \\
	\vattrib{Matthew}{7:8}
\end{verse}
That's a promise from God, just knock!!! We're not talking about knocking and then assuming no one
is home. We are talking about pressing in.
\begin{verse}
	\V{1} My son, if you receive my words and treasure up my commandments with you, \V{2} making your
	ear attentive to wisdom and inclining your heart to understanding; \V{3} yes, if you call out for
	insight and raise your voice for understanding, \V{4} if you seek it like silver and search for
	it as for hidden treasures, \V{5} then you will understand the fear of the Lord and find the
	knowledge of God. \\
	\vattrib{Proverbs}{2:1-5}
\end{verse}
What does it take? Not a quick knock. Look at all the action terms used there, . Put that kind of
effort into knowing his word. But what do we do? We just depend on other people to study it for us.
\begin{verse}
	\V{24} And he said to them, "Pay attention to what you hear: with the measure you use, it
	will be measured to you, and still more will be added to you." \\
	\vattrib{Mark}{4:24}
\end{verse}
A believer grows in proportion to the understanding and application of what they have already
received. You've received the word of God? What do you do with it? Pay attention to what you have
received, to the measure that you have opened and received it, that is how it will be measured to you.
\begin{verse}
	\V{25} For to the one who has, more will be given, and from the one who has not, even what he has
	will be taken away. \\
	\vattrib{Mark}{4:25}
\end{verse}
Whoever hasn't pondered, or opened his heart to what he has receieved, even what he has received
will be taken away. Is it possible for someone to go to church and end up with a blank slate? It
definitely is.
\begin{verse}
	\V{2} All these things my hand has made, and so all these things came to be, declares the Lord.
	But this is the one to whom I will look: he who is humble and contrite in spirit and trembles
	at my word. \\
	\vattrib{Isaiah}{66:2}
\end{verse}
\subsection{The Kingdom of God in Power}
The Kingdom of God does not show up like man's kingdoms. So here Jesus uses simple pictures, to
speak of the Kingdom.
\begin{verse}
	\V{26} He also said, "This is what the kingdom of God is like. A man scatters seed on the ground.
	V{27} Night and day, whether he sleeps or gets up, the seed sprouts and grows, though he does not
	know how. \V{28} All by itself the soil produces grain | first the stalk, then the head, then the
	full kernel in the head." \\
	\vattrib{Mark}{4:26-28}
\end{verse}
This parable is about the mystery of growth. The Kingdom of God grows according to the purpose
and plan of God. Which is why it grows in a mysterious way. The growth of the seed takes place
by the hand of God, it is not man's will. It is God who gives the increase, you just need to be
faithful. If God wants you to be faithful to one person, you better be faithful to one person.
\begin{verse}
	"\V{29} As soon as the grain is ripe, he puts the sickle to it, because the harvest has come." \\
	\vattrib{Mark}{4:29}
\end{verse}
There is an end. There is going to come a time where the harvest will come, and we'll enter
another age.
\begin{verse}
	\V{30} Again he said, "What shall we say the kingdom of God is like, or what parable shall we
	use to describe it? \V{31} It is like a mustard seed, which is the smallest of all seeds on
	earth." \\
	\vattrib{Mark}{4:30-31}
\end{verse}
For us, we are living 2000 years down the road. The mustard seed plant is big and mature.
We've seen for a couple of thousand years this mature plant Christianity has planted.
\begin{verse}
	\V{32} "Yet when planted, it grows and becomes the largest of all garden plants, with such big
	branches that the birds can perch in its shade." \\
	\vattrib{Mark}{4:32}
\end{verse}
If we follow the idea of expositional constancy, we can see that birds have a negative connotation.
What some bible teachers think is that Jesus was not only promising that the kingdom of God was
going to start small and grow large, but its also going to be influenced and infested at some
point with worldly and even demonic things. Having the benefit of history, is that what has happened?
\begin{verse}
	\V{33} With many similar parables Jesus spoke the word to them, as much as they could understand.
	\V{34} He did not say anything to them without using a parable. But when he was alone with his
	own disciples, he explained everything. \V{35} That day when evening came, he said to his
	disciples, "Let us go over to the other side." \V{36} Leaving the crowed behind, they took him
	along, juts as he was, in the boat. There were also other boats with him. \V{37} A furious squall
	came up, and the waves broke over the boat, so that it was nearly swamped. \V{38} Jesus was in the
	stern, sleeping on a cushion. The disciples woke him and said to him, "Teacher, don't you care if we
	drown?" \V{39} He got up, rebuked the wind and said to the waves, "Quiet! Be still!" Then the
	wind died down and it was completely calm. \V{40} He said to the disciples, "Why are you so
	afraid? Do you still have no faith?" \V{41} They were terrified and asked each other, "Who is this?
	Even the wind and the waves obey him!" \\
	\vattrib{Mark}{4:33-41}
\end{verse}
They had not only been listening to Jesus teach all day long. And seen the glory of God manifest all
day long. How much of the power of God do you have to see before you have faith that you are in the
midst of the power of God? Do we see God as being asleep during in the time of our storm?
\begin{verse}
	\V{23} Awake, Lord! Why do you sleep? Rouse yourself! Do not reject us forever. \V{24} Why do
	you hide your face and forget our misery and oppression? \\
	\vattrib{Psalms}{44:23-24}
\end{verse}
\begin{verse}
	\V{27} Peace I leave with you; my peace I give to you. I do not give to you as the world gives. Do
	not let your hearts be troubled and do not be afraid. \\
	\vattrib{John}{14:27}
\end{verse}

\section{Chapter 5}
\subsection{The Dynamic Of Faith}
\begin{verse}
	\V{1} They went across the lake to the region of the Gerasenes. \V{2} When Jesus got out of the
	boat, a man with an impure spirit came from the tombs to meet him. \V{3} This man lived in the
	tombs, and no one could bind him anymore, not even with a chain. \V{4} For he had often been
	chained hand and foot, but he tore the chains apart and broke the irons on his feet. No one
	was strong enough to subdue him. \V{5} Night and day among the tombs and in the hills he would
	cry out and cut himself with the stones. \V{6} When he saw Jesus from a distance, he ran and fell
	on his knees in front of him. \V{7} He shouted at the top of his voice, "What do you want with me,
	Jesus, Son of the Most High God? In God's name don't torture me!" \V{8} For Jesus had said to him,
	"Come out of this man, you impure spirit!" \V{9} The Jesus asked him, "What is your name?"
	"My name is Legion," he replied, "for we are many." \V{10} And he begged Jesus again and again not
	to send them out of the area. \V{11} A large herd of pigs was feeding on the nearby hillside. \V{12}
	The demons begged Jesus, "Send us among the pigs; allow us to go into them." \V{13} He gave them
	permission, and the impure spirits came out and went into the pigs. The herd, about two thousand in
	number, rushed down the steep bank into the lake and were drowned. \V{14} Those tending the pigs
	ran off and reported this in the town and countryside, and the people went out to see what had
	happened. \V{15} When they came to Jesus, they saw the man who had been possessed by the legion of
	demons, sittng there, dressed and in his right mind; and they were afraid. \V{16} Those who had
	seen it told the people what had happened to the demon-possessed man | and told about
	the pigs as well. \V{17} Then the people began to plead with Jesus to leave their region. \V{18}
	As Jesus was getting into the boat, the man who had been demon-possessed begged to go with him. \V{19}
	Jesus did not let him, but said, "Go home to your own people and tell them how much the Lord has
	done for you, and how he has had mercy on you." \V{20} So the man went away and began to tell in
	the Decapolis how much Jesus had done for him. And all the people were amazed. \\
	\vattrib{Mark}{5:1-20}
\end{verse}
Demons know who Jesus is. Men aren't so sure. This demon possessed man lived among the tombs, and had
some kind of supernatural strength. He would be naked and cry out night and day. He would also cut himself
with stones. So where's the faith here? In this story it seems like the power of God manifests without
any expression of faith. It semes like there is only God's desire to accomplish his own will.
\begin{verse}
	\V{20} Now to him who is able to do immeasurably more than all we ask or imagine, according to his
	power that is at work within us, \V{21} to him be the glory in the church and in Christ Jesus
	throughout all generations, for ever and ever! Amen. \\
	\vattrib{Ephesians}{3:20-21}
\end{verse}
He's abundantly able to do so much more than \textbf{what we ask.} He's even able to be able to do
abundantly more than what we think. Aren't you glad that what we think doesn't limit our God.
\begin{verse}
	\V{21} When Jesus had again crossed over by boat to the other side of th elake, a large crowd
	gathered around him while he was by the lake. \V{22} Then one of the synagogue leaders, named
	Jairus, came, and when he saw Jesus, he fell at his feet. \V{23} He pleaded earnestly with him,
	"My little daughter is dying. Please come and put your hands on her so that she will be healed
	and live." \V{24} So Jesus went with him. A large crowd followed and pressed around him. \V{25}
	And a woman was there who had been subject to bleeding for twelve years. \V{26} She had suffered
	a great deal under the care of many doctors and had spent all she had, yet instead of getting better
	she grew worse. \V{27} When she had heard about Jesus, she came up behind him in the crowd and
	touched his cloak, \V{28} because she thought, "If I just touch his clothes, I will be healed."
	\V{29} Immediately her bleeding stopped and she felt in her body that she was freed from her
	suffering. \V{30} At once Jesus realized that power had gone out from him. He turned around in the
	crowd and asked, "Who touched my clothes?" \V{31} "You see the people crowding against you," his
	disciples answered, "and yet you can ask, 'Who touched me?'" \V{32} But Jesus kept looking around to
	see who had done it. \V{33} Then the woman, knowing what had happened to her, came and fell at his
	feet and, trembling with fear, told him the whole truth. \V{34} He said to her, "Daughter, your
	faith has healed you. Go in peace and be freed from your suffering." \V{35} While Jesus was still
	speaking, some people came from the house of Jairus, the synangogue leader. "You daughter is dead,"
	they said. "Why bother the teacher anymore?" \V{36} Overhearing what they said, Jesus told him,
	"Don't be afraid; just believe." \V{37} He did not let anyone follow him except Peter, James and
	John the brother of James. \V{38} When they came to the home of the synagogue leader, Jesus saw a
	commotion, with people crying and wailing loudly. \V{39} He went in and said to them, "Why all this
	commotion and wailing? The child is not dead but asleep." \V{40} But they laughed at him. After
	he put them all out, he took the child's father and mother and the disciples who were with him,
	and went in where the child was. \V{41} He took her by the hand and said to her, "Talitha koum!"
	(which means "Little girl, I say to you, get up!"). \V{42} Immediately the girl stood up and began
	to walk around (she was twelve years old). At this they were completely astonished. \V{43} He gave
	strict orders not to let anyone know about this, and told them to give her something to eat. \\
	\vattrib{Mark}{5:21-43}
\end{verse}
Where did she come up with the idea of touching his garment? Well it came from her own heart.
And she did it and she was immediately healed. That was her faith. Different people, different faith.
Jesus met them at the point of their faith. Jairus wanted Jesus to lay hands on his daughter.
God will meet you where you are, God just looks at the heart that is reaching out to him in
faith. Those who laughed at the face of God's power were put outside. The laughers and the
mockers got to see nothing.
\begin{verse}
	\V{5} Trust in the Lord with all your heart and lean not on your own understanding; in all your
	ways submit to him, and he will make your paths straight. \\
	\vattrib{Proverbs}{3:5-6}
\end{verse}
Don't fixate on what you know and what you understand. God is able to do abundantly more than
what you think.

\section{Chapter 6}
\subsection{Faith and Unbelief}
\begin{verse}
	\vattrib{Mark}{6:1-6}
\end{verse}
\begin{verse}
	\vattrib{Luke}{7:9}
\end{verse}
One is an example of unbelief, and the other is that he marveled at great faith. Jesus' hometown had
heard all the things that he had done and his ministry. When just before he had been just chased
out of this synagogue for saying that he is the Messiah. But Jesus could still do no mighty works
among them because of their unbelief.
\begin{verse}
	\vattrib{Isaiah}{8:14-15}
\end{verse}
The people of his hometown still did not belief in him, and took offense at him. They were offended
by who he is, because we call him saviour. They refused to honour him. Their unbelief limited what
Jesus could do among them.
\begin{verse}
	\vattrib{Matthew}{13:58}
\end{verse}
So why aren't we seeing the same miracles today? Some people say that miracles died out at the end
of the 'apostolic era'. Aren't they just doing the same thing as the people of Nazareth? They never
once considered the possibility, that they are the problem. Jesus even rebuked his believers for
unbelief, and their hard hearts.
\begin{verse}
	\vattrib{Hebrews}{3:12}
\end{verse}
And there's a question that's asked in the bible
\begin{verse}
	\vattrib{Luke}{18:8}
\end{verse}
Where's my faith? What's going on inside my heart? What should we do then in a society of increasing
unbelief?
\begin{itemize}
	\item Guard your heart
	\item Feed your heart/build your faith
	\item Pray for God's love
	\item Be prepared to share the hope you have in Jesus
\end{itemize}
\end{document}
