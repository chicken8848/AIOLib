\documentclass{article}
\usepackage{verse}
\usepackage{bibleref}
\usepackage{attrib}

% Custom Macros
\def\V#1 {\(^{#1}\)}
\def\vattrib#1#2{\attrib{\bibleverse{#1}(#2)}}

\title{The Gospel of Mark}
\author{Joshua's Notes}

\begin{document}
\maketitle
\section*{Introduction} % (fold)
\label{sec:introduction}
Mark is a very succint, quick, gospel. He focuses on Jesus as a worker, the deeds of Jesus. That's the
significance of this book. Mark was not part of the 12 disciples. He was a young boy during the life
and ministry of Jesus. The early church of Jerusalem met in Mark's home. He accompanied the apostle
Paul and Barnabas on the first part of their missionary outreach. Mark later became a sort of assistant
to the apostle Peter. And most of these events were an eyewitness testimony of Peter, because Mark
was so young during the time.

\section{Chapter 1}
\subsection{John Appeared, Baptizing in the Wilderness}
\begin{verse}
	\begin{altverse}
		\V{1}
		The beginning of the gospel of Jesus Christ, the Son of God. \\
	\end{altverse}
	\vattrib{Mark}{1:1}
\end{verse}
This gospel of which we are concerned, is the only everlasting good news. Without Jesus, our mentality
becomes very materialistic, happiness becomes
the end. There would be no meaning to life, and everyone tries to be happy, so they will end up doing
bad stuff.

\begin{verse}
	\V{2}
	As it is written in Isaiah the prophet,
	\begin{quote}
		"Behold, I send my messenger before your face, \\
		Who will prepare your way, \\
		\V{3} The voice crying in the wilderness:\\
		Prepare the way of the Lord,\\
		Make his paths straight,"
	\end{quote}
	\vattrib{Mark}{1:2}
\end{verse}
Mark is referring to the prophecy given by Isaiah.
\begin{verse}
	\V{1} Comfort, comfort my people, says your God. \\
	\V{2} Speak tenderly to Jerusalem, \\
	and cry to her \\
	that her warfare is ended, \\
	that her iniquity is pardoned, \\
	that she has received from the Lord's hand \\
	double for all her sins. \\
	\V{3} A voice cries: \\
	In the wilderness prepare the way of the Lord; \\
	make straight in the desert a highway for our God.\\
	\vattrib{Isaiah}{1:1-3}
\end{verse}
This is one of many Old Testament prophecies that actually has a dual fulfillement in that this prophecy
relates to the first coming of Jesus, and also the second coming of Christ. And that is many case with
many Old Testament prophecies. Mark is pointing out the fact that Isaiah prophesied of one who would
come and prepare the way. And we're told who that is.

\begin{verse}
	\V{4} John appeared, baptizing in the wilderness and proclaiming a baptism of repentance for the
	forgiveness of sins. \V{5} And al the country of Judea and all Jerusalem were going out to him
	and were being baptized by him in the river Jordan, confessing their sins. \\
	\vattrib{Mark}{1:4-5}
\end{verse}
This is a huge move of God, because people were gathering, for one purpose: to confess their sins.
It's not easy to get people to get together to talk about their sin, even more so repenting of it.
It's a miracle! What does it mean to repent though? Repent means to have a change of mind, to change
your actions based on your change of mind. What is this move of God going to do? How does repentance
prepare the way of the Lord? We have to go to another gospel account to answer this question.

\subsubsection{Why is repenting important?}
\begin{verse}
	\V{29} When all the people heard this, and the tax collectors too, they declared God just, having
	been baptized with the baptism of John, \V{30} but the Pharisees and the lawyers rejecetd the
	purpose of God for themselves, not having been baptized by him. \\
	\vattrib{Luke}{7:29-30}
\end{verse}
The power of repentance is that it opens our hearts to hear God, and respond to him in a positive
manner. If there is a hardness of heart that refuses to repent, the Word of God just kind of bounces
off us. The people who received Jesus and understood him, they prepared by repenting.

\subsection{The Spirit of Elijah}
\begin{verse}
	\V{6} Now John was clothed with camel's hair and wore a leather belt around his waist and ate
	locusts and wild honey. \\
	\vattrib{Mark}{1:6}
\end{verse}
Elijah wore much of the same clothes. But more interesting than that is that, in Luke, when the
angel prophecied to Zechariah about John the baptist, that John would go forth in the spirit and
power of Elijah. But John never performed a miracle, yet the Bible tells us he goes forth in the
spirit and power of Elijah, which is shown in his clothing.

\begin{verse}
	\V{7} And he preached, saying,
	\begin{quote}
		"After me comes he who is mightier than I, \\
		the strap of whose sandals I am not worthy to stoop down and untie."
	\end{quote}
	\vattrib{Mark}{1:7}
\end{verse}
To the jew particular statement elevated Jesus and humbled John. The teachers in Judaism used to say that
a teacher could ask pretty much anything of his student except to help him off with his shoes. That
was supposed to be beneath any student. John comes along and to speak of the greatness of the one
who comes after and says I am not worthy to take off his shoes. He further goes on to contrast his
ministry and continues with verse 8.

\begin{verse}
	\begin{quote}
		\V{8} "I have baptized you with water, but he will baptize you with the Holy Spirit."
	\end{quote}
	\vattrib{Mark}{1:8}
\end{verse}
John is immersing you in water, and he's preparing you to meet Jesus. But Jesus is going to immerse
you in God. Jesus is going to bring God into your life. Baptism doesn't change your life, the spirit
of God does. Living in him. Baptism won't change the condition of your heart, but being filled with
the Holy Spirit will.
\subsection{The Baptism of Christ}
\begin{verse}
	\V{9} In those days Jesus came from Nazareth of Galilee and was baptized by John in the Jordan.
	\V{10} And when he came up out of the water, immediately he saw the heavens being torn open and the
	Spirit descending on him like a dove. \V{11} And a voice came from heaven, "You are my beloved Son;
	with you I am well pleased." \\
	\vattrib{Mark}{1:9-11}
\end{verse}
Jesus came to the midst of the Jordan river filled with sinful people, and identified with them.
Jesus would hang on the cross after 3 years for these sinners. The bible says he became sin, that
we might become the righteousness of God.
\subsection{Driven into the Wilderness}
\begin{verse}
	\V{12} The Spirit immediately drove him out into the wilderness. \V{13} And he was in the wilderness
	forty days, being tempted by Satan. And he was with the wild animals, and the angels were ministering
	to him. \\
	\vattrib{Mark}{1:12}
\end{verse}
Jesus had 40 days of temptation, even though we only learn of 3 particular temptations. Imagine being
tempted for 40 days. This is Jesus going through what we go through. Most of our lives we have to deal
with temptation and the enemy trying to bring us further from God. Jesus came to drink the entire
cup of the condition of man.
\subsubsection{The sympathy of Jesus}
\begin{verse}
	\V{15} For we do not have a high priest who is unable to sympathize with our weaknesses, but one
	who in every respect has been tempted as we aree, yet without sin.
\end{verse}
\vattrib{Hebrews}{4:15}
He not only experienced what we experienced, but also gave the way not to fall into temptation. With
the word of God, and faith, he came through with victory. Which can now become our victory. To be victorious
over sin, its not a method, we look to Jesus. We enter into his victory. The source is a person, not
a few steps. (As we see on facebook or whatever)

\subsection{Jesus Proclaims the Gospel}
\begin{verse}
	\V{14} Now after John was arrested, Jesus came into Galilee, proclaiming the gospel of God, \V{15} and
	saying, "The time is fulfilled, and the kingdom of God is at hand; repent and believe in the gospel." \\
	\vattrib{Mark}{1:14-15}
\end{verse}
Jesus spoke repentance as well. He also said the time was fulfilled, the coming of Messiah!
\subsection{Follow Jesus}
\begin{verse}
	\V{16} Passing alongside the Sea of Galilee, he saw Simon and Andrew the brother of Simon casting
	a net into the sea, for they were fishermen. \V{17} And Jesus said to them, "Follow me, and I will
	make you become fishers of men." \V{18} And immediately they left their nets and followed him. \\
	\vattrib{Mark}{1:16-18}
\end{verse}
This is actually the second time Jesus has spoken to them, and Mark does not record the first time.
But now that he has started his Galilean ministry, he called them and they immediately followd him.
And Mark loves to use this word "immediately", uses it about 40 times. Peter had a family, and they
dropped everything and immediately followed Jesus, even though he was probably the only breadwinner.
When Jesus calls us, its nay or yay. Its impossible to do both. Its a decision, are you going to follow
Jesus?
\begin{verse}
	\V{19} And going on a little farther, he saw James the son of Zebedee and John his brother, who were in
	their boat mending the nets. \V{20} And immediately he called them, and they left their father
	Zebedee in the boat with the hired servants and followed him.
\end{verse}
\vattrib{Mark}{1:19-20}
And he says, "I'm going to make you fishers of men." We all talents, what he's saying is take the
things he has given you and follow him to the kingdom of God. These guys were fishermen, Jesus said
"Fine, let's go fish, but we're gonna fish for different things. You're a bulider, let's go build
different things. You're a banker, let's go invest in different things." What you are in the world,
bring it into the kingdom, that has eternal scope and value. New goals for old powers.
\subsection{The Authority Of Jesus}
\begin{verse}
	\V{21} And they went into Capernaum, and immediately on the Sabbath he entered the synagogue and
	was teaching. \V{22} And they were astonished at his teaching, for he taught them as one who had
	authority, and not as the scribes. \\
	\vattrib{Mark}{1:21-22}
\end{verse}
The Jewish teachers back in those days, they would spend a lot of time quoting other teachers. And
sometimes give opposing viewpoints of different Rabbis, but Jesus didn't do that, he just spoke
by the authority of his own word. We forget sometimes when we get into debates, that truth is not a
debate process, truth is a person. When we get into debates, it's important to remember not to get
caught up in the conversation of opinion, but rather let the words of Jesus to speak for itself.
Because Jesus spoke with authority. Don't get lost in other people's and your own opinion. It's not
just important to tell others what Jesus said, it is also important to pinpoint it for others.
\begin{verse}
	\V{23} And immediately there was in their synagogue a man with an unclean spirit. And he cried
	out, \\
	\vattrib{Mark}{1:23}
\end{verse}
The word unclean is a borrowing from an Old Testament concept of ceremonially unclean. But here it
just sort of means possessed by demonic spirit. Mark uses language in the greek, which is similar to
the language Paul uses when he describes a believer who is indwelled by the Holy Spirit. Be possessed
by the Lord's spirit, it will bring you freedom, peace, and eternal life.

\subsubsection{How does one get posssessed by a demon?}
The Bible doesn't really explain it.
Is it possible for me as a Christian, to be possessed by a demon? Stay invested in God's spirit,
the two cannot dwell together. You never even see or hear of the mention of an exorcism for a believer
in the Bible. But rather a testing of faith, and to repent if you sin. You can't blame demonic
possession.
\begin{verse}
	\V{18} We know that everyone who has been born of God does not keep on sinning, but he who was
	born of God protects him, and the evil one does not touch him.
	\vattrib{John}{5:18}
\end{verse}
"He" refers to Jesus. When it comes to believers, its a hands off approach when it comes to the work
of the enemy. That does not mean we won't have spiritual battles, or testing.
\begin{verse}
	\V{8} Be sober-minded; be watchful. Your adversary the devil prowls around like a roaring lion, seeking
	someone to devour. \V{9} Resist him, firm in your faith, knowing that the same kinds of suffering
	are being experienced by your brotherhood throughout the world. \\
	\vattrib{IPeter}{5:8-9}
\end{verse}
So the reality of satanic activity real according scripture.
\begin{verse}
	\V{12} For we do not wrestle against flesh and blood, but against the rulers, against the authorities,
	against the cosmic powers over this present darkness, against the spiritual forces of evil in the
	heavenly places. \\
	\vattrib{Ephesians}{6:12}
\end{verse}
Battles are a lot in life. But as a believer, possession is not in the cards for you
\subsection{The Authority Of Jesus}
\begin{verse}
	\V{24} "What have you to do with us, Jesus of Nazareth? Have you come to destroy us? I know who
	you are | the Holy One of God." \\
	\vattrib{Mark}{1:24}
\end{verse}
The demons would often speak through the host to expose Jesus, and then start freaking out. Jesus
always commanded to be silent, not to tell who he was. But why? The Jews wanted to be free of Roman
leadership, Jesus knew that the Zealots would jump at the opportunity to go against the Roman
Empire.
\begin{verse}
	\V{25} But Jesus rebuked him saying,
	\begin{quote}
		"Be silent, and come out of him!"
	\end{quote}
	\V{26} And the unclean spirit, convulsing him and crying out with a loud voice, came out of him. \\
	\vattrib{Mark}{1:26}
\end{verse}
This guy has authority, his exorcism is short. Which is different from other exorcisms.
\begin{verse}
	\V{27} And they were all amazed, so that they questioned among themselves, saying,
	\begin{quote}
		"What is this? A new teaching with authority! He commands even the unclean spirits, and they
		obey him."
	\end{quote}
	\V{28} And at once his fame spread everywhere throughout all the surrounding region of Galilee.\\
	\vattrib{Mark}{1:27-28}
\end{verse}
\begin{verse}
	\V{29} And immediately he left the synagogue and entered the house of Simon and Andrew, with James
	and John. \V{30} Now Simon's mother-in-law lay ill with a fever, and immediately they told him
	about her. \V{31} And he came and took her by the hand and lifted her up, and the fever left her, and
	she began to serve them. \\
	\vattrib{Mark}{1:29-31}
\end{verse}
What does she do with the new life and the healing that was given her? She just starts
serving him. God wants to restore to us what we have lost, and serve him with it. Just like this
woman. What has God restored in your life? Serve the Lord with all of your heart, give it to him.
\begin{verse}
	\V{32} That evening at sundown they brought to him all who were sick or oppressed by demons.
	\V{33} And the whole city was gathered together at the door. \V{34} And he healed many who were
	sick with various diseases, and cast out many demons. And he would not permit the demons to speak,
	because they knew him. \\
	\vattrib{Mark}{1:32-34}
\end{verse}
This is God's compassion, trust in the compassion of God. We do not need to pray him into compassion.
His heart is moved by people with faith in him.
\subsubsection{The Demons know Jesus}
\begin{verse}
	\V{19} You believe that God is one; you do well. Even the demons believe | and shudder!
\end{verse}
You believe in God? Even demons believe in God. Do not rest in the superficiality of knowing Jesus,
or even just believing that he exists. Do not stop at the superficial, it has to be personal.
\begin{verse}
	\V{12} But to all who did receive him, who believed in his name, he gave the right to become children
	of God. \\
	\vattrib{John}{1:12}
\end{verse}
Communion is a picture of receiving Jesus. It is not intellectual, but it is of partaking, and now
it is inside of you. Have you received him as your saviour? It is going past religion, to relationship.
\subsection{The Inconvenience of Following God}
\begin{verse}
	\V{35} And rising very early in the morning, while it was still dark, he departed and went out to
	a desolate place, and there he prayed. \V{36} And Simon and those who were with him searched for him,
	\V{37} and they found him and said to him,
	\begin{quote}
		"Everyone is looking for you."
	\end{quote}
	\V{38} And he said to them,
	\begin{quote}
		"Let us go on to the next towns, that I may preach there also, for that is why I came out."
	\end{quote}
	\V{39} And he went throughout all Galilee, preaching in their \\>
	synangogues and casting out demons. \\
	\vattrib{Mark}{1:35-39}
\end{verse}
A long day of ministry, goes into the night, Jesus probably lays down to get some rest and before
the sun even rose, he went out to pray. This shows his heart toward prayer. Some of us wait for
prayer to be convenient, it is not convenient. It is never convenient to pray. It's hard work,
its challenging, and you may have to give something up to do it. For Jesus, he gave up his sleep,
which we hang onto so tightly. Prayer for Jesus meant sacrifice too. No one likes getting up when
its dark.
\end{document}
