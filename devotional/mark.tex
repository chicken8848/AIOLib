\documentclass{article}
\usepackage{verse}
\usepackage{bibleref}
\usepackage{attrib}

% Custom Macros
\def\V#1 {\(^{#1}\)}
\def\vattrib#1#2{\attrib{\bibleverse{#1}(#2)}}

\title{The Gospel of Mark}
\author{Joshua's Notes}

\begin{document}
\maketitle
\section*{Introduction} % (fold)
\label{sec:introduction}
Mark is a very succint, quick, gospel. He focuses on Jesus as a worker, the deeds of Jesus. That's the
significance of this book. Mark was not part of the 12 disciples. He was a young boy during the life
and ministry of Jesus. The early church of Jerusalem met in Mark's home. He accompanied the apostle
Paul and Barnabas on the first part of their missionary outreach. Mark later became a sort of assistant
to the apostle Peter. And most of these events were an eyewitness testimony of Peter, because Mark
was so young during the time.

\section{Chapter 1}
\subsection{Verse 1}
\begin{verse}
	\begin{altverse}
		\V{1}
		The beginning of the gospel of Jesus Christ, the Son of God. \\
	\end{altverse}
	\vattrib{Mark}{1:1}
\end{verse}
This gospel of which we are concerned, is the only everlasting good news. Without Jesus, our mentality
becomes very materialistic, happiness becomes
the end. There would be no meaning to life, and everyone tries to be happy, so they will end up doing
bad stuff.

\subsection{Verse 2-3}
\begin{verse}
	\V{2}
	As it is written in Isaiah the prophet,
	\begin{quote}
		"Behold, I send my messenger before your face, \\
		Who will prepare your way, \\
		\V{3} The voice crying in the wilderness:\\
		Prepare the way of the Lord,\\
		Make his paths straight,"
	\end{quote}
	\vattrib{Mark}{1:2}
\end{verse}
Mark is referring to the prophecy given by Isaiah.
\begin{verse}
	\V{1} Comfort, comfort my people, says your God. \\
	\V{2} Speak tenderly to Jerusalem, \\
	and cry to her \\
	that her warfare is ended, \\
	that her iniquity is pardoned, \\
	that she has received from the Lord's hand \\
	double for all her sins. \\
	\V{3} A voice cries: \\
	In the wilderness prepare the way of the Lord; \\
	make straight in the desert a highway for our God.
\end{verse}
\vattrib{Isaiah}{1:1-3}
This is one of many Old Testament prophecies that actually has a dual fulfillement in that this prophecy
relates to the first coming of Jesus, and also the second coming of Christ. And that is many case with
many Old Testament prophecies. Mark is pointing out the fact that Isaiah prophesied of one who would
come and prepare the way. And we're told who that is.

\subsection{Verse 4-5}
\begin{verse}
	\V{4} John appeared, baptizing in the wilderness and proclaiming a baptism of repentance for the
	forgiveness of sins. \V{5} And al the country of Judea and all Jerusalem were going out to him
	and were being baptized by him in the river Jordan, confessing their sins.
\end{verse}
\vattrib{Mark}{1:4-5}
This is a huge move of God, because people were gathering, for one purpose: to confess their sins.
It's not easy to get people to get together to talk about their sin, even more so repenting of it.
It's a miracle! What does it mean to repent though? Repent means to have a change of mind, to change
your actions based on your change of mind. What is this move of God going to do? How does repentance
prepare the way of the Lord? We have to go to another gospel account to answer this question.

\subsubsection{Why is repenting important?}
\begin{verse}
	\V{29} When all the people heard this, and the tax collectors too, they declared God just, having
	been baptized with the baptism of John, \V{30} but the Pharisees and the lawyers rejecetd the
	purpose of God for themselves, not having been baptized by him.
\end{verse}
\vattrib{Luke}{7:29-30}
The power of repentance is that it opens our hearts to hear God, and respond to him in a positive
manner. If there is a hardness of heart that refuses to repent, the Word of God just kind of bounces
off us. The people who received Jesus and understood him, they prepared by repenting.

\subsection{Verse 6}
\begin{verse}
	\V{6} Now John was clothed with camel's hair and wore a leather belt around his waist and ate
	locusts and wild honey.

\end{verse}
\vattrib{Mark}{1:6}
Elijah wore much of the same clothes. But more interesting than that is that, in Luke, when the
angel prophecied to Zechariah about John the baptist, that John would go forth in the spirit and
power of Elijah. But John never performed a miracle, yet the Bible tells us he goes forth in the
spirit and power of Elijah, which is shown in his clothing.

\subsection{Verse 7}
\begin{verse}
	\V{7} And he preached, saying,
	\begin{quote}
		"After me comes he who is mightier than I, \\
		the strap of whose sandals I am not worthy to stoop down and untie."
	\end{quote}
\end{verse}
\vattrib{Mark}{1:7}
To the jew particular statement elevated Jesus and humbled John. The teachers in Judaism used to say that
a teacher could ask pretty much anything of his student except to help him off with his shoes. That
was supposed to be beneath any student. John comes along and to speak of the greatness of the one
who comes after and says I am not worthy to take off his shoes. He further goes on to contrast his
ministry and continues with verse 8.

\subsection{Verse 8}
\begin{verse}
	\begin{quote}
		\V{8} I have baptized you with water, but he will baptize you with the Holy Spirit.
	\end{quote}
\end{verse}
\vattrib{Mark}{1:8}
John is saying

\end{document}
