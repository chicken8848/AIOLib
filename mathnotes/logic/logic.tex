\setlength\parindent{0pt}
\documentclass{article}
\usepackage{amsmath}
\usepackage{amsfonts}
\usepackage{amssymb}
\usepackage{txfonts}
\usepackage{MnSymbol}
\usepackage[a4paper, total={6in, 8in}]{geometry}


\title{Introduction to Mathematical Logic}
\author{Joshua John Lee Shi Kai}
\begin{document}

\maketitle
\tableofcontents

\newpage

\section*{Introduction}
The rules of logic give precise meaning to mathematical statements.
These rules are used to distinguish between valid and invalid mathematical arguments.
The reader will learn how to understand and how to construct correct
mathematical arguments, this is an introduction to logic.
\section{Propositions}
A proposition is a declarative sentence that is either true or false,
but not both.
\subsection[Proposition definition]{Example 1}
Consider the following:
\begin{center}
	\begin{tabular}{ll}
		Propositions:    & 1. $1 + 1 = 2$                           \\
		                 & 2. There are infinitely many twin primes \\
		                 & 3. Toronto is the capital of Canada      \\
		                 &                                          \\
		Not Propositions & 1. What time is it?                      \\
		                 & 2. $x + y = 2$                           \\
		                 & 3. $x + y = z$                           \\
	\end{tabular}
\end{center}
Sentence 1 is not a proposition because it is not declarative.
Sentences 2 and 3 are not propositions because they are neither true nor
false. Note that we can turn 2 and 3 into propositions by assigning
numbers to the variables.

\noindent \\ Many mathematical statements are constructed by combining
one ore more propositions. New propositions, called \textbf{compound propositions},
are formed from existing proposition susing logical operators

\subsection[Negation of a proposition]{Definition 1}
Let $p$ be a proposition. The \textit{negation of} $p$,
denoted by $\neg{p}$, is the statement:
\begin{quote}
	"It is not the case that $p$."
\end{quote}
The truth value of $\neg{p}$ is the opposite of the truth value of $p$.

\subsection*{Example 2}
Find the negation to the following
\begin{quote}
	"Today is Friday"
\end{quote}
\textit{Solution:} The negation is simply:
\begin{quote}
	"Today is not Friday,"
\end{quote}
or
\begin{quote}
	"It is not the case that today is Friday."
\end{quote}
\subsection[Conjunction of propositions]{Definition 2}
Let $p$ and $q$ be propositions. The \textit{conjunction} of $p$ and $q$,
denoted by $p \wedge q$, is the proposition "$p$ and $q$".
It is true when both $p$ and $q$ are true and false otherwise.
\subsection*{Example 3}
Find the conjuction of the propositions $p$ and $q$, where

\begin{center}
	\begin{tabular}{ll}
		$p$ : & "Today is Friday"     \\
		      &                       \\
		$q$ : & "It is raining today" \\
	\end{tabular}
\end{center}
\textit{Solution:} $p \wedge q$ is simply the proposition
"Today is Friday and it is raining today."
\subsection[Disjunction of propositions]{Definition 3}
Let $p$ and $q$ be propositions. The \textit{disjunction} of $p$ and $q$,
denoted by $p \vee q$, is the proposition "$p$ or $q$".
It is false when both $p$ and $q$ are false and true otherwise.
\subsection*{Example 4}
What is the disjunction of the propositions $p$ and $q$ where $p$ and $q$ are defined as is in
Example 3?

\noindent \\ \textit{Solution:} The disjunction of $p$ and $q$, $p \vee q$ is simply the proposition
\begin{quote}
	"Today is Friday or it is raining today."
\end{quote}
This proposition is true on any day that is either a Friday or a rainy day (including rainy Fridays).
It is only false on days that are not Fridays when it also does not rain.
\subsection[Exclusive or]{Definition 4}
Let $p$ and $q$ be propositions. The \textit{exclusive or} of $p$ and $q$, denoted by $p \oplus q$,
is the proposition that is true when exactly one of $p$ and $q$ is true and is false otherwise.
\subsection[Conditional Statements]{Definition 5}
Let $p$ and $q$ be propositions. The \textit{conditional statement} $p \Rightarrow q$ is the proposition
"if $p$, then $q$." The conditional statement $p \Rightarrow q$ is false when $p$ is true and $q$ is false, and true
otherwise. In the conditional statement $p \Rightarrow q$, $p$ is called the \textit{hypothesis/antecedent/premise}
and $q$ is called the \textit{conclusion/consequence}.

\noindent Here are some ways to express conditionals

\begin{itemize}
	\item "p implies q"
	\item "p only if q"
	\item "a sufficient condition for q is p"
	\item "q is necessary for p"
	\item "q follows from p"
	\item "if p, then q"
\end{itemize}

\noindent \\ The statement $p \Rightarrow q$ is called a conditional statement because $p \Rightarrow q$ asserts
that $q$ is true on the condition that $p$ holds. A conditional statement is also called an \textbf{implication}.

\noindent \\ A useful way to understand the truth value of a conditional statemnt is to think of an
obligation or contract. For example, a pledge of a politician,
\begin{quote}
	"If I am elected, then I will lower taxes."
\end{quote}
If the politician is elected, voters would expect this politician to lower taxes. Furthermore, if the policitian
is not elected, then voters will not have any expectation that this person will lower taxes, even thought this person
might have sufficient influence to cause those in power to lower taxes. It is only when the politician is elected
but does not lower taxes that the people will say the politician has broken the campaign pledge. This last scenario
corresponds to the case where $p$ is true but $q$ is false.
\subsection*{Example 5}
What is the value of the variable $x$ after the statement
\begin{quote}
	\textbf{if} $2+2=4$, \textbf{then} $x_{n} \equal x_{n-1}+1$
\end{quote}
if $x = 0$ before this statement is encountered?

\noindent \\ \textit{Solution:} Because $2+2 = 4$ is true, the statement $x_n = x_{n-1}+1$ is executed.
Hence $x$ has the value $0+1 = 1$ after this statement is encountered.

\subsection[Converse, contrapositive, inverse]{Definition 6}
\begin{center}
	\begin{tabular}{|l|l|l|l|}
		\hline
		Proposition       & Converse          & Contrapositive                & Inverse                       \\
		\hline
		$p \Rightarrow q$ & $q \Rightarrow p$ & $\neg{q} \Rightarrow \neg{p}$ & $\neg{p} \Rightarrow \neg{q}$ \\
		\hline
	\end{tabular}
\end{center}
\begin{itemize}
	\item The contrapositive always has the same truth value as the proposition.
	\item Neither the convers nor the inverse have the same truth value as the proposition.
	\item When two compound propositions always have the same truth values we call them \textbf{equivalent}.
	\item Propostion and contrapositive are equivalent. The converse and inverse are also logically equivalent. (More later)
\end{itemize}
\subsection*{Example 6}
What are the contrapositive, the convers, and the inverse of the conditional statement
\begin{quote}
	"The home team wins whenever it is raining."
\end{quote}
\textit{Solution:} Because "$q$ whenever $p$" is one of the ways to express the conditional statement
$p \Rightarrow q$, the original statement can be rewritten as
\begin{quote}

	"If it is raining, then the home team wins."

\end{quote}
Consequently, the contrapositive of this conditional statement is
\begin{quote}
	"If the home team does not win, then it is not raining."
\end{quote}
The converse is
\begin{quote}
	"If the home team wins, then it is raining."
\end{quote}
The inverse is
\begin{quote}
	"If it is not raining, then the home team does not win."
\end{quote}
Only the contrapositive is equivalent to the original statement

\subsection[Biconditionals]{Definition 7}
Let $p$ and $q$ be propositions. The \textit{biconditional statement} $p \Leftrightarrow q$ is the
proposition "$p$ if and only if $q$". The biconditional statement $p \Leftrightarrow q$ is true when
$p$ and $q$ have the same truth value and false otherwise. Biconditional statements are also called
\textit{bi-implications}.

\noindent \\ Here are some ways express $p \Leftrightarrow q$:

\begin{itemize}
	\item "p is necessary and sufficient for q"
	\item "if p then q, and conversely"
	\item "p iff q"
\end{itemize}

\noindent The last way of expressing the biconditional statement uses the abbreviation "iff"
for "if and only if". Note that $p \Leftrightarrow q$ has exactly the same truth value as $(p \Rightarrow q) \wedge (q \Rightarrow p)$.
\subsection*{Example 7}
Let $p$ be the statement "You can take the flight " and let $q$ be the statement "You buy a ticket."
Then $p \Leftrightarrow q$ is the statement
\begin{quote}
	"You can take the flight if and only if you buy a ticket."
\end{quote}
\subsection{Precedence of Logical Operators}
\begin{center}
	\begin{tabular}{c|c}
		Operator          & Precedence \\
		$\neg$            & 1          \\
		$\wedge$          & 2          \\
		$\vee$            & 3          \\
		$\Rightarrow$     & 4          \\
		$\Leftrightarrow$ & 5          \\
	\end{tabular}
\end{center}
\section{Propositional Equivalences}
\subsection*{Introduction}
An important type of step used in a mathematical argument is the replacement of a statement with another statement with the same truth value.
Because of this, motheds that produce propositions with the same truth value as a given compound proposition are used extensively in the construction
of mathematical arguments. A compound proposition is any statement with two or more propositions combined with logical operators.
\subsection*{Definition 1}
A compound proposition that is always true, no matter what the truth values of the propositions that occur in it, is called a \textit{tautology}.
A compound proposition that is always false is called a \textit{contradiction}. A compound proposition that is neither a tautology nor a contradiction
is called a \textit{contingency}. Tautologies and contradictions are often used in mathematical reasoning.

\subsection{Logical Equivalences}
\subsection*{Definition 2}
The compound propositions $p$ and $q$ are called \textit{logically equivalent} if $p \Leftrightarrow q$ is a tautology.

\noindent \\ One way to show two compound propositions are equivalent is to use a truth table. The compound propositions of
$p$ and $q$ are equivalent if and only if the columns giving their truth values agree.

\subsection*{Example 1}
Show that $\neg (p \vee q)$ and $\neg p \wedge \neg q$ are logically equivalent.

\noindent \\ \textit{Solution:} Because the truth values for the compound propositions agree for all possible
combination of the truth values of $p$ and $q$, it follows that $\neg (p \vee q) \Leftrightarrow \neg p \wedge \neg q$
is a tautology.

\begin{center}
	\begin{tabular}{|c|c|c|c|c|c|c|}
		\hline
		$p$ & $q$ & $p \vee q$ & $\neg(p \vee q)$ & $\neg p$ & $\neg q$ & $\neg p \wedge \neg q$ \\
		\hline
		T   & T   & T          & F                & F        & F        & F                      \\
		T   & F   & T          & F                & F        & T        & F                      \\
		F   & T   & T          & F                & T        & F        & F                      \\
		F   & F   & F          & T                & T        & T        & T                      \\
		\hline
	\end{tabular}
\end{center}

\subsection*{Logical Equivalences Table}
\begin{center}
	\begin{tabular}{|l | l|}
		\hline
		Equivalence                                                          & Name                \\
		\hline
		$p \wedge T \Leftrightarrow p$                                       & Identity Laws       \\
		$p \vee F \Leftrightarrow p$                                         &                     \\
		\hline
		$p \vee T \Leftrightarrow T$                                         & Domination Laws     \\
		$p \wedge F \Leftrightarrow F$                                       &                     \\
		\hline
		$p \vee p \Leftrightarrow p$                                         & Idempotent Laws     \\
		$p \wedge p \Leftrightarrow p$                                       &                     \\
		\hline
		$\neg (\neg p) \Leftrightarrow p$                                    & Double negation law \\
		\hline
		$p \vee q \Leftrightarrow q \vee p$                                  & Commutative laws    \\
		$p \wedge q \Leftrightarrow q \wedge p$                              &                     \\
		\hline
		$(p \vee q) \vee r \Leftrightarrow p \vee (q \vee r)$                & Associative laws    \\
		$(p \wedge q) \wedge r \Leftrightarrow p \wedge (q \wedge r)$        &                     \\
		\hline
		$p \vee (q \wedge r) \Leftrightarrow (p \vee q) \wedge (p \vee r)$   & Distributive laws   \\
		$p \wedge (q \vee r) \Leftrightarrow (p \wedge q) \vee (p \wedge r)$ &                     \\
		\hline
		$\neg (p \vee q) \Leftrightarrow \neg p \wedge \neg q$               & De Morgan's Laws    \\
		$\neg (p \wedge q) \Leftrightarrow \neg p \vee \neg q$               &                     \\
		\hline
		$p \vee (p \wedge q) \Leftrightarrow p$                              & Absorption laws     \\
		$p \wedge (p \vee q) \Leftrightarrow p$                              &                     \\
		\hline
		$p \vee \neg p \Leftrightarrow T$                                    & Negation laws       \\
		$p \wedge \neg p \Leftrightarrow F$                                  &                     \\
		\hline
	\end{tabular}

\end{center}

\subsection{Constructing New Logical Equivalences}
Using the laws we can construct additional logical equivalances. The reason for this is that a proposition
in a compound proposition can be replaced by a compound proposition that is logically equivalent to it without changing
thet truth value of the original compound proposition.

\subsection*{Example 2}
Show that the two compound propositions are equivalent.
\begin{enumerate}
	\item $p \Rightarrow q$
	\item $\neg p \vee q$
\end{enumerate}

\noindent \textit{Solution:} We will establish this equivalence by developing a series of logical equivalences.
\begin{center}
	\begin{tabular}{l l l l}
		$p \Rightarrow q$ & $\Leftrightarrow$ & $\neg (p \wedge \neg q)$    & by definition of conditional statement \\
		                  & $\Leftrightarrow$ & $\neg p \vee \neg (\neg q)$ & by the second De Morgan Law            \\
		                  & $\Leftrightarrow$ & $\neg p \vee q$             & by the double negation law
	\end{tabular}

\end{center}

\section{Predicates and Quantifiers}
\subsection*{Introduction}
In this section we will introduce a more powerful type of logic called \textbf{predicate logic}.
It enables us to reason and explore relationships in between objects. To understand predicate logic,
we first introduce the concept of a predicate. Then the notion of quantifiers, which help us
reason with statements that assert that a certain property holds for all objects of a certain
type and with statements that assert the existence of an object with a particular property.
\subsection{Predicates}
Consider the statement involving a variable, namely

\begin{quote}
	"$x > 3$"
\end{quote}

\noindent The statement "$x$ is greater than 3" has two parts. The first part, the variable $x$, is the subject
of the statement. The sceond part, the \textbf{predicate}, "is greater than 3", refers to a property that
the subject of the statement can have. We can denote the statement "$x > 3$" by $P(x)$,
where P denotes the predicate "is greater than 3" and $x$ is the variable. $P(x)$ is also said to be the
propositional function at $x$. Once a value has been assigned to the variable $x$, the statement $P(x)$ becomes
a proposition and has a truth value.

\subsection*{Example 1}
Let $Q(x,y)$ denote the statement "$x = y + 3$" What are the truth values of the propositions $Q(1,2)$ and $Q(3,0)$?

\noindent \\ \textit{Solution:} $Q(1,2)$ is the statement "$1 = 2+3$" which is false. $Q(3,0)$ is the statement
"$3=0+3$" which is true.

\subsection{Quantifiers}
There is a way, to create a proposition from a propositional function. This way is called \textit{quantification}.
Quantification expresses to the extent to which a predicate is true over a range of elements. We will focus on two types
of quantification: universal quantification, which tells us that a predicate is true for every element under consideration, and
existential quantification, that tells us that there is one or more elements under consideration for which the predicate is true.
The area of logic that deals with predicates and quantifiers is called the \textbf{predicate calculus}.

\subsection*{Universal Quantifiers}
Many mathematical statements assert that a property is true for all values of a variable in a particular domain,
often referred to as the domain. Such a statement is expressed using universal quantification. The meaning of the
universal quantification of $P(x)$ changes when we change the domain. Thus the domain must always be specified when
a universal quantifier is used.

\subsection*{Definition 1}
The \textit{universal quantification} of $P(x)$ is the statement
\begin{quote}
	"$P(x)$ for all values of $x$ in the domain"
\end{quote}
The notation $\forall xP(x)$ denotes the universal quantification of $P(x)$. Here $\forall$ is called the \textbf{universal quantifier.}
We read $\forall xP(x)$ as "for all $xP(x)$" or "for every $xP(x)$." An element for which $P(x)$ is false is called a \textbf{counter example} of
$\forall xP(x)$.

\subsection*{Example 2}
Let $P(x)$ be the statement "$x+1>x$" What is the truth value of the quantification $\forall xP(x)$, where the domain consists of all real numbers?

\noindent \\ \textit{Solution:} Because $P(x)$ is true for all real numbers $x$, the quantification $\forall xP(x)$ is true.

\subsection*{Existential Quantifiers}
Many mathematical statements assert that there is an element with a certain property. Such statements are expressed using existential quantification.
With existential quantification we form a propostion that is true if and only if $P(x)$ is true for at least one value of $x$ in the domain.

\subsection*{Definition 2}
The \textit{existential quantification} of $P(x)$ is the proposition
\begin{quote}
	"There exists an element $x$ in the domain such that $P(x)$"
\end{quote}
We use the notation $\exists xP(x)$ for the existantial quantification of $P(x)$. Here $\exists$ is called the \textbf{existential quantifier}.

\subsection*{Example 3}
What is the truth value of $\exists xP(x)$, where $P(x)$ is the statement "$x^2 > 10$" and the universe of discourse consists of the
positive integers not exceeding 4?

\noindent \\ \textit{Solution:} Because the domain is ${1,2,3,4}$, the proposition $\exists xP(x)$ is the same as the disjunction
\begin{quote}
	$P(1) \vee P(2) \vee P(3) \vee P(4)$
\end{quote}

\noindent Because $P(4)$ which is the statement "$4^2 > 10$" is true, it follows that $\exists xP(x)$ is true.

\subsection*{Precedence of Quantifiers}
Quantifiers have a higher precedence than all logical operators from propositional calculus. For example, $\forall xP(x) \vee Q(x)$ means
$(\forall xP(x)) \vee Q(x)$ rather than $\forall x(P(x) \vee Q(x))$.

\begin{center}
	\begin{tabular}{|l|l|l|}
		\hline
		Statement       & When True?                               & When False?                         \\
		\hline
		$\forall xP(x)$ & $P(x)$ is true for every $x$             & Exists an $x$ where $P(x)$ is false \\
		\hline
		$\exists xP(x)$ & There is an $x$ for which $P(x)$ is true & $P(x)$ is false for every $x$       \\
		\hline
	\end{tabular}
\end{center}

\subsection{Binding Variables}
When a quantifier is used on the variable $x$, we say that this occurence of the variable is \textbf{bound}.
An occurence of a variable that is not bound is called \textbf{free}. All variables of a propositional function must be
bound or set equal to a value to turn it into a proposition.

\subsection{Logical Equivalences Involving Quantifiers}
The notion of logical equivalences can be extended to involve predicates and quantifiers.
\subsection*{Definition 3}
Statements involving predicates and quantifiers are \textit{logically equivalent} if and only if they have the same truth value
no matter which predicates are substituted into these statements and which domain of discourse is used for the variables in these propositional functions.
We use the notation $S \equiv T$ to indicate that two statements involving predicates and quantifiers are logically equivalent.

\subsection{Negating Quantified Expressions}
\begin{center}
	\begin{tabular}{|l|l|}
		\hline
		Negation             & Equivalent Statement  \\
		\hline
		$\neg \exists xP(x)$ & $\forall x \neg P(x)$ \\
		$\neg \forall xP(x)$ & $\exists x \neg P(x)$ \\
		\hline
	\end{tabular}
\end{center}

\subsection*{Example 4}
Consider these statements, of which the first three are premises and the fourth is a valid conclusion.

\begin{enumerate}
	\item	"All hummingbirds are richly colored"
	\item "No large birds live on honey"
	\item	"Birds that do not live on honey are dull in color"
	\item	"Humming birds are small"
\end{enumerate}

\noindent Let $P(x), Q(x), R(x), S(x)$ be the statements "$x$ is a hummingbird", "$x$ is large," "$x$ lives on honey," and
"$x$ is richly colored."
Assuming the domain consists of all birds, express the statements in the argument using quantifiers and propositional functions.

\noindent \\ \textit{Solution:} We can express the statements in the argument as
\begin{enumerate}
	\item $\forall x (P(x) \Rightarrow S(x))$
	\item $\neg \exists x (Q(x) \wedge R(x))$
	\item $\forall x(\neg R(x) \Rightarrow \neg S(x))$
	\item $\forall x(P(x) \Rightarrow \neg Q(x))$
\end{enumerate}
\subsection{Propositional Satisfiability}
A compound is \textbf{satisfiable} if there is an assignment of truth values to its variables that makes it true.
When no such assignments exist, that is, when the compound proposition is false for all assignments of truth values to its variables,
the compound proposition is \textbf{unsatisfiable}. A math-y way to understand this is that a compound proposition is unsatisfiable if and only
if its negation is true for all assignments of truth values to the variables, that is, if and only if its negation is a tautology.

\noindent When we do find a particular assignment of truth values that makes a compound proposition true, we have shown that it is satisfiable, this
assignment is then called a \textbf{solution} of this particular problem.
\subsection*{Example 3}
Determine whether each of the compound propositions are satisfiable
\begin{enumerate}
	\item $(p \vee \neg q) \wedge (q \vee \neg r) \wedge (r \vee \neg p)$
	\item $(p \vee \neg q) \wedge (q \vee \neg r) \wedge (r \vee \neg p) \wedge (p \vee q \vee r) \wedge (\neg p \vee \neg q \vee \neg r)$
\end{enumerate}

\noindent \textit{Solution:} 1 is true when all $p, q, r$ have the same truth values, it is thus satisfiable.
For 2, we note that the first 3 compounds must have the same truth values for $p, q, r$ to be true, but for the next 2 compounds, at least one must be false.
However, these conditions are contradictory. Hence it is unsatisfiable.

\subsection{An Application to Satisfiability}
\subsection*{Sudoku Puzzle}
To encode a Sudoku puzzle, let $p(i,j,n)$ denote the propostition that is true when the number $n$ is in the cell in the $i$th row and the $j$th column.
There are $9 \cdot 9 \cdot 9 = 729$ such propositions, as $i$, $j$, and $n$ all range from 1 to 9.
\newline Given a particular Sudoku puzzle, we begin by encoding each of the given values. Then, we construct compound propositions that assert that every
row contains every number, every column contains every number, every $3 \times 3$ block contains every number, and each cell contains no more than one number.
It follows, that the Sudoku puzzle is solved by finding an assignment of truth values to the 729 propositions $p(i,j,n)$ with each $i, j, n,$ ranging from 1 to 9 that makes
the conjunction of all these compound propositions true.

\noindent \\ Here we note that:
\begin{enumerate}
	\item $\forall iP_{i} \Leftrightarrow P_{1} \wedge P_{2} \wedge P_{3} \wedge \cdots \wedge P_{i} \Leftrightarrow \wedge_{i=1}^{n}P_{i}$
	\item $\exists iP_{i} \Leftrightarrow P_{1} \vee P_{2} \vee P_{3} \vee \cdots \vee P_{i} \Leftrightarrow \vee_{i=1}^{n}P_{i}$
	\item $P(i,j,n) \Leftrightarrow$ "Row $i$ and column $j$ contain number $n$"
\end{enumerate}

\noindent We start by breaking down our problem into some propositions
\begin{enumerate}
	\item $q_{1} \Leftrightarrow$ "Every row must contain every number" \\ $\Rightarrow \forall i \forall n \exists j P(i,j,n)$ \\
	      $\Rightarrow \wedge_{i=1}^{9} \wedge_{n=1}^{9} \vee_{j=1}^{9} P(i,j,k)$
	\item $q_{2} \Leftrightarrow$ "Every column must contain every number" \\ $\Rightarrow \forall j \forall n \exists i P(i,j,n)$ \\
	      $\Rightarrow \wedge_{j=1}^{9} \wedge_{n=1}^{9} \vee_{i=1}^{9} P(i,j,k)$
	\item $q_{3} \Leftrightarrow$ "Each of the $3 \times 3$ block must contain every number" \\
	      $\Rightarrow \forall r \forall c \forall n \exists m \exists s P(3r+m, 3c+s, n)$ \\
	      $\Rightarrow \wedge_{r=0}^{2} \wedge_{c=0}^{2} \wedge_{n=1}^{9} \vee_{m=1}^{3} \vee_{s=1}^{3}P(3r+m, 3c+s, n)$
	\item $q_{4} \Leftrightarrow$ "No cell contains more than 1 number" \\
	      $\Rightarrow \forall i \forall j \forall n \forall n^{'}(n \nequal n^{'} \wedge P(i,j,n) \Rightarrow \neg P(i,j,n^{'}))$ \\
	      $\Rightarrow \wedge_{i=1}^9 \wedge_{j=1}^9 \wedge_{n=1}^9 \wedge_{n^{'}=1}^9 (n \nequal n^{'} \wedge P(i,j,n) \Rightarrow \neg P(i,j,n^{'}))$
\end{enumerate}

\noindent Solving the Sudoku puzzle is the same as solving the satisfiablity problem for the statement $q_{1} \wedge q_{2} \wedge q_{3} \wedge q_{4}$.
We will then need to check $2^{729}$ possible combinations of $i, j, n$ to the proposition $P(i,j,n)$.

\section{Additional Exercises}
\subsection*{Exercise 1}
Show that $\neg$ and $\wedge$ form a functionally complete collection of logical operators. [Hint: First use a De Morgan law
		to show that $p \vee q$ is equivalent to $\neg (\neg p \wedge \neg q)$]

\subsection*{Exercise 2}
Establish these logical equivalences, where $x$ does not occur as a free variable in $A$. Assume that the domain is non-empty.
\begin{enumerate}
	\item $\forall x(P(x) \Rightarrow A) \Leftrightarrow \exists xP(x) \Rightarrow A$
	\item $\exists x(A \Rightarrow P(x)) \Leftrightarrow \forall xP(x) \Rightarrow A$
\end{enumerate}
\subsection*{Exercise 3}
Show that $(p \wedge q) \Rightarrow (p \vee q)$ is a tautology.





\end{document}
