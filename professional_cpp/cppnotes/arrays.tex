\setlength\parindent{0pt}
\documentclass{article}
\usepackage{amsmath}
\usepackage{amsfonts}
\usepackage{amssymb}
\usepackage{txfonts}
\usepackage{MnSymbol}
\usepackage{array}
\usepackage{tabularx}
\usepackage{fancyvrb}
\usepackage[a4paper, total={6in, 8in}]{geometry}

\begin{document}
\section{Arrays}
\textit{Arrays} hold a series of values, all of the same type, each of which can be assessed by a position
in the array. In C++ the size of the array \textbf{must} be provided when declared. You cannot give a
variable as the size | it must be a constant, or a \textit{constant expression}.
\begin{verbatim}
    int myArray[3];
    myArray[0] = 0;
    myArray[1] = 0;
    myArray[2] = 0;
\end{verbatim}
You can also get the same effect by using the following
\begin{verbatim}
   int myArray[3] = {0};
\end{verbatim}
Note that this is only possible if you want to initialize all values to zero. For example, the following
fills only the first element in the array with the value 2 and the rest of the elements with value 0.
\begin{verbatim}
   int myArray[3] = {2};
\end{verbatim}
An array can also be initialized with an initializer list, in which case the compiler can deduce the
size of the array automatically.
\begin{verbatim}
   int arr[] = {1,2,3,,4}; // Compiler creates an array of 4 elements.
\end{verbatim}
This shows a one-dimensional array, but C++ allows multi-dimensional arrays.
\begin{verbatim}
   char ticTacToeBoard[3][3];
   ticTacToeBoard[1][1] = 'o';
\end{verbatim}
\section{std::array}
The arrays discussed in the previous section come from C, and still work in C++. However, C++ has a
special type of fixed-size container called std::array, defined in the <array> header file.

\noindent \\ There are a number of advantages in using std::arrays instead of C-style arrays.
\begin{enumerate}
	\item They always know their own size
	\item Do not automatically get cast to a pointer to avoid certain types of bugs
	\item Have iterators to easily loop over the elements.
\end{enumerate}
\begin{verbatim}
#include <iostream>
#include <array>
using namespace std;
int main() {
    array<int,3> arr = {9,8,7};
    cout << "Array size = " << arr.size() << endl;
    cout << "Element 2 = " << arr[1] << endl;
    return 0
}
\end{verbatim}
Note that both C-style and std::arrays have a fixed size, which should be known at compile time.
They cannot grow or shrink at run time. If you want an array with a dynamic size, it's recommended
to use std::vector, explained later. A vector immediately grows in size when you add new elements to
it.
\end{document}
