\setlength\parindent{0pt}
\documentclass{article}
\usepackage{amsmath}
\usepackage{amsfonts}
\usepackage{amssymb}
\usepackage{txfonts}
\usepackage{MnSymbol}
\usepackage{array}
\usepackage{tabularx}
\usepackage{fancyvrb}
\usepackage[a4paper, total={6in, 8in}]{geometry}

\begin{document}
\section{Conditionals}
\textit{Conditionals} let you execute code based on whether or not something is true.
As shown in the following sections, there are three main types of conditionals
in C++.
\subsection{if/else statements}
The most common is the if/else statement
\begin{verbatim}
if (i > 4) {
// Do something
} else if (i>2) {
// Do something else
} else {
// Do soemthing else
}
\end{verbatim}
The expression between the parathesis must be a boolean value.
\section{switch statements}
The switch statement is an alternate syntax for performing actions based on the value of an expression.
In C++ switch statements, the expression must be of an integral type or of a type that is convertible
to integral type, and must be compared to constants. Each value represents a "case". Then the code
following the case is executed until a break statement is reached. You can also provide a default case,
which is matched if none of the other cases match.
\begin{verbatim}
switch(menuItem) {
    case OpenMenuItem:
        // Code to open a file
        break;
    case SaveMenuItem:
        // Code to save a file
        break;
    default:
        // Code to give an error message
        break;
}
\end{verbatim}
A switch statement can always be converted into if/else statements. The previous statement can be
converted as follows:
\begin{verbatim}
if (menuItem == OpenMenuItem) {
    // Code to open a file
} else if (menuItem == SaveMenuItem) {
    // Code to save a file
} else {
    // Code to give an error message
}
\end{verbatim}
If a case section does not have a break statement, the code for that case section is executed first,
followed by a \textit{fallthrough}, executing the code for the next case section whether or not
that case matches. This can be a source of bugs, but is sometimes useful. One example is to have
a single case section that is executed for several different cases. For example,
\begin{verbatim}
   case ColorDarkBlue;
   case ColorBlack:
      // Code to execute for both a dark blue or black background colour.
      break;
   case ColorRed:
      // Code to execute for a red background colour
      break;
\end{verbatim}
\subsection{The conditional operator}
C++ has one operator that takes three arguments, known as the \textit{ternary operator}. It is used
as a shorthand if/else statement. The following code will output "yes" if the variable is greater than
2, and "no" otherwise.
\begin{verbatim}
   std::cout << ((i > 2) ? "yes" : "no");
\end{verbatim}
Unlike an if statement or a switch statemnt, the conditional operator doesn't execute code blocks based
on the result. Instead it is used \textit{within} code, as shown as the preceding example. In a sense,
it is an operator (like $+$ and $-$) as opposed to a true conditional.
\subsection{Logical Evaluation Operators}
\begin{center}
	\begin{tabularx}{|\raggedleft X| X| X|}
		\textbf{OP} & \textbf{DESCRIPTION} & \textbf{USAGE} \\
	\end{tabularx}
\end{center}
\end{document}
