\setlength\parindent{0pt}
\documentclass{article}
\usepackage{amsmath}
\usepackage{amsfonts}
\usepackage{amssymb}
\usepackage{txfonts}
\usepackage{MnSymbol}
\usepackage{array}
\usepackage{tabularx}
\usepackage[a4paper, total={6in, 8in}]{geometry}


\title{C++ Notes}
\author{Joshua John Lee Shi Kai}
\begin{document}

\maketitle
\tableofcontents

\newpage
\section{Preprocessor Directives}
The following table shows some of the most common preprocessor directives
\begin{center}
	\begin{tabularx}{\textwidth}{
			| >{\raggedright\arraybackslash}X
			| >{\raggedright\arraybackslash}X
			| >{\raggedright\arraybackslash}X |
		}
		\hline
		\textbf{PREPROCESSOR DIRECTIVE} & \textbf{FUNCTIONALITY}                                                                                                                                                           & \textbf{COMMON USES}                                                                                                                                                                                                                                              \\
		\hline
		\#include [file]                & The specified file is inserted into the code at the location of the directive                                                                                                    & Almost always used to include header files so that code can make use of functionality defined elsewhere                                                                                                                                                           \\
		\hline
		\#define [key] [value]          & Every occurence of the specified key is replaced with the specified value.                                                                                                       & Often used in C to define a constant value or a macro. C++ provides better mechanisms for constants and macros. Macros are often dangerous so use them cautiously.                                                                                                \\
		\hline
		\#ifdef [key] \newline
		\#endif \newline
		\#ifndef [key] \newline
		\#endif                         & Code within the ifdef ("if defined") or ifndef ("if not defined") blocks are conditionally included or omitted based on whether the specified key has been defined with \#define & Used most frequently to protect against circlar includes. Each include file starts with a \#ifndef checking the absence of a key, followed by defining that key. The include file ends with a \#endif. This prevents the file from being included multiple times. \\
		\hline
		\#pragma [xyz]                  & xyz varies from compiler to compiler. Often allows the programmer to display a warning or error if the directive is reached during preprocessing                                 & See examples                                                                                                                                                                                                                                                      \\
		\hline
	\end{tabularx}
\end{center}



\end{document}
