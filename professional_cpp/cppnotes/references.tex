\setlength\parindent{0pt}
\documentclass{article}
\usepackage{amsmath}
\usepackage{amsfonts}
\usepackage{amssymb}
\usepackage{txfonts}
\usepackage{MnSymbol}
\usepackage{array}
\usepackage{tabularx}
\usepackage{fancyvrb}
\usepackage[a4paper, total={6in, 8in}]{geometry}

\begin{document}
\section{References}
Attaching \& to a type indicates that the variable is a reference. Here are two implementations of an
addOne() function. The first will have no effect on the variable that is passed in because it is passed
by value. The second uses a reference and thus changes the original variable.
\begin{verbatim}
         void addOne(int i){
                i++; // Has no real effect because this is a copy of the original
         }
         void addOne(int& i) {
                i++; // Actually changes the original variable
         }
\end{verbatim}
The syntax for the call to the addOne() function with an integer reference is no different than if
the function just took an integer.
\begin{verbatim}
         int myInt = 7;
         addOne(myInt);
\end{verbatim}
\end{document}
