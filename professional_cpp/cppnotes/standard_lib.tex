\setlength\parindent{0pt}
\documentclass{article}
\usepackage{amsmath}
\usepackage{amsfonts}
\usepackage{amssymb}
\usepackage{txfonts}
\usepackage{MnSymbol}
\usepackage{array}
\usepackage{tabularx}
\usepackage{fancyvrb}
\usepackage[a4paper, total={6in, 8in}]{geometry}

\begin{document}
\section{The Standard Library}
Don't reinvent certain classes, use the standard library. They are tuned for high performance and are
probably better than your implementation.
\subsection{std::vector}
The \texttt{vector} replaces the concept of C style arrays with a much more flexible and safer
mechanism. As a user, you need not worry about memory management, as the \texttt{vector} will
automatically allocate enough memory to hold its elements. A \texttt{vector} is dynamic, meaning
that elements can be added and removed at run time. To make it easy to loop over the contents
of containers, the standard library provides a concept called \textit{iterators}.
\begin{verbatim}
   #include <string>
   #include <vector>
   #include <iostream>
   #include <iterator>
   using namespace std;
   int main() 
   {
      // Create a vector of strings, using uniform initialization
      vector<string> myVector = {"A first string", "A second string"};
      // Add some strings to the vector using push_back
      myVector.push_back("A third string");
      myVector.push_back("The last string in the vector");
      // Print the elemnts using a range-based for loop
      for (const auto& str : myVector)
          cout << str << endl;
      // Iterate over the elemnts in the vector and print them once more
      for (auto iterator = cbegin(myVector);
          iterator != cend(myVector); ++iterator) {
          cout << *iterator << endl;
          }
          return 0;
   }
\end{verbatim}
\end{document}
