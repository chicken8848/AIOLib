\setlength\parindent{0pt}
\documentclass{article}
\usepackage{amsmath}
\usepackage{amsfonts}
\usepackage{amssymb}
\usepackage{txfonts}
\usepackage{MnSymbol}
\usepackage{array}
\usepackage{tabularx}
\usepackage{fancyvrb}
\usepackage[a4paper, total={6in, 8in}]{geometry}

\begin{document}
\section{Strings in C}
C++ has a \texttt{string} type defined in the \texttt{<string>} header file, and that you
can use a C++ \texttt{string} almost like a basic type. The \texttt{string} type lives in the
\texttt{std} namespace.
\begin{verbatim}
std::string myString = "Hello, World";
cout << "The value of myString is " << myString << endl;
\end{verbatim}

Unlike C, C++ provides a string type as a first-class data type. C++ contains several functions
from the C language that operate on strings, defined in the \texttt{<cstring>} header. These functions
do not handle memory allocation. For example, the \texttt{strcpy()} functions takes two strings as
parameters. It copies the second string onto the first, whether it fits or not. The following code
attempts to build a wrapper around \texttt{strcpy()} that allocates the correct amount of memory
and returns the result, instead of taking in an already allocated string. It uses the \texttt{strcpy()}
function to obtain the length of the string. The caller is responsible for freeing the memory allocated
by \texttt{copyString()}.
\begin{verbatim}
  char* copyString(const char* str)
  {
    char* result = new char[strlen(str) + 1];
    strcpy(result, str);
    return result;
  }
\end{verbatim}

\subsection{String Literals}
You've probably seen something like this
\begin{verbatim}
   cout << "hello" << endl;
\end{verbatim}
In the preceding line, \texttt{"hello"} is a \textit{string literal} because it is written as a value,
not as variable. Memory associated with a string literal is in a read-only part of memory. Allowing
the compiler to optimize memory usage by reusing references to equivalent string literals. This is
called \textit{literal pooling.}

\subsection{The C++ string class}
C++ provides a much-improved implementation of the concept of a string as part of the Standard Library.
In C++ \texttt{std::string} is a class that supports many of the same functionalities as the \texttt{<cstring>}
functions, but takes care of memory allocation for you.

\begin{verbatim}
  string A("12");
  string B("34");
  string C;
  C = A + B; // C will become "1234"
\end{verbatim}


Another problem with C strings is that you cannot use \texttt{==} to compare them. With C++,
\texttt{==, !=, <, >} etc, are all overloaded to work on the actual string characters.

\subsection{Numeric Conversions}

The \texttt{std} namespace includes a number of helper functions to convert numerical values into
strings and vice versa.

\begin{verbatim}
string to_string(int val);
string to_string(unsigned val);
string to_string(long val);
string to_string(unsigned long val);
string to_string(long long val);
string to_string(unsigned long long val);
string to_string(float val);
string to_string(double val);
string to_string(long double val);
string to_string(int stoi(const string& str, size_t *idx=0, int base=10));
string to_string(long double val);
string to_string(long stol(const string& str, size_t *idx=0, int base=10));
string to_string(unsigned long stoul(const string& str, 
                                    size_t *idx=0, int base=10));
string to_string(long long stoll(const string& str, size_t *idx=0, int base=10));
string to_string(unsigned long long stoull(const string& str, 
                                          size_t *idx=0, int base=10));
string to_string(float stof(const string& str, size_t *idx=0));
string to_string(double stod(const string& str, size_t *idx=0));
string to_string(long double stold(const string& str, size_t *idx=0));
\end{verbatim}
\end{document}
