\setlength\parindent{0pt}
\documentclass{article}
\usepackage{amsmath}
\usepackage{amsfonts}
\usepackage{amssymb}
\usepackage{txfonts}
\usepackage{MnSymbol}
\usepackage{array}
\usepackage{tabularx}
\usepackage{fancyvrb}
\usepackage[a4paper, total={6in, 8in}]{geometry}

\begin{document}
\section{Structs}
\textit{Structs} let you encapsulate one or more existing types into a new type. The classic example
of a struct is a database record. If you are building a personnel systetm to keep track of employee
information, you will need to store the first initial, last initial, employee number, and salary for
each employee. A struct that contains all of this information is shown in the employeestruct.h header
file that follows:
\begin{verbatim}
struct Employee {
    char firstInitial;
    char lastInitial;
    int employeeNumber;
    int salary;
};
\end{verbatim}
A variable declared with type Employee will have all of these fields built-in. The individual fields
of a struct can be accessed by using the "." operator. The example that follows creates and then
outputs the record for an employee:
\begin{verbatim}
#include <iostream>
#include "employeestruct.h"
using namespace std;
int main()
{
    // Create and populate an employee.
    Employee anEmployee;
    anEmployee.firstInitial = 'M';
    anEmployee.lastInitial = 'G';
    anEmployee.employeeNumber = 42;
    anEmployee.salary = 80000;
    // Outpyut the values of an employee
    cout << "Employee: " << anEmployee.firstInitial  << 
                            anEmployee.lastInitial << endl;
    cout << "Number: " << anEmployee.employeeNumber << endl;
    cout << "Salary: $" << anEmployee.salary << endl;
    return 0;
}
  
\end{verbatim}
\end{document}
